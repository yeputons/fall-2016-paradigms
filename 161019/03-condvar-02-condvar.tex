\subsection{События}

\begin{frame}[fragile]{Новый примитив}
	Введём примитив \t{Event} с двумя методами:
	\begin{itemize}
		\item \t{e.wait()} "--- усыпляет поток.
		\item \t{e.notify()} "--- будит уснувший поток.
	\end{itemize}
	\begin{tabular}{p{0.45\linewidth}p{0.45\linewidth}}
		\centering
\begin{minted}{cpp}
// Producer
while (true) {
  int data = get_data();
  q.push(data);
  e.notify();
}
\end{minted}
&
\begin{minted}{cpp}
// Consumer
while (true) {
  if (!q.empty()) {
    process_data(q.pop());
  } else {
    e.wait();
  }
}
\end{minted}
	\end{tabular}
	Есть ли проблемы в коде выше?
	\pause
	Проблемы есть.
\end{frame}

\begin{frame}{Ну вы поняли}
	\begin{center}
		\includegraphics[scale=0.4]{race-condition-everywhere.jpg}
	\end{center}
\end{frame}

\begin{frame}{Чуть подробнее}
	\begin{enumerate}
		\item Consumer проверяет \t{!global\_queue.empty()}
		\item Producer добавляет данные.
		\item Producer вызывает \t{e.notify()}, а будить некого.
		\item Consumer вызывает \t{e.wait()} и засыпает навечно.
	\end{enumerate}
	Что делать?
\end{frame}

\begin{frame}{Первый подход}
	\begin{itemize}
		\item Можно сказать, что если в момент вызова \t{e.notify()} никто не спит, то будет разбужен следующий попытающийся уснуть.
		\item Другими словами, у \t{Event} теперь есть состояние: просигналили или нет.
		\item \t{e.notify()} "--- устанавливает флаг <<просигналили>> и будит все потоки.
		\item \t{e.wait()} "--- ждёт, пока флаг установят (или не ждёт, если уже установлен) и сбрасывает его.
		\item Решает задачу producer-consumer.
		\item Используются в Windows API.
	\end{itemize}
	Однако:
	\begin{itemize}
		\item Дополнительное состояние вносит сложность "--- за ним надо следить и добавлять инвариант.
		\item В pthread не входят и под Linux обычно не используются.
	\end{itemize}
\end{frame}

\begin{frame}[fragile]{Второй подход: добавим мьютексов?}
	\begin{tabular}{p{0.45\linewidth}p{0.45\linewidth}}
		\centering
\begin{minted}{cpp}
// Producer
while (true) {
  int data = get_data();
  pthread_mutex_lock(&m);
  q.push(data);
  e.notify();
  pthread_mutex_unlock(&m);
}
\end{minted}
&
\begin{minted}{cpp}
// Consumer
while (true) {
  pthread_mutex_lock(&m);
  if (!q.empty()) {
    process_data(q.pop());
  } else {
    e.wait();
  }
  pthread_mutex_unlock(&m);
}
\end{minted}
	\end{tabular}
	Теперь race condition отсутствует.
	\pause
	Зато есть deadlock: producer не может ничего писать, пока consumer спит.
\end{frame}

\subsection{Условные переменные}

\begin{frame}{Условные переменные}
	\begin{itemize}
		\item Нам нужна атомарная операция <<отпусти мьютекс и жди события>>.
		\item Такой примитив синхронизации в pthread (и вообще много где) называется \textit{условная переменная} (conditional variable).
		\item Смысл: условная переменная "--- это способ оповещать потоки о \textit{возможном} изменении некоторого \textit{условия}, защищённого мьютексом.
		\item Ожидание пассивное, ресурсы CPU не тратятся.
		\item На каждое условие создаётся условная переменная.
		\item Поток, изменивший условие, может разбудить либо все ожидающие потоки (\t{signal}), либо один случайный (\t{broadcast}).
		\item Бывают spurious wakeup "--- система иногда может разбудить ждущий поток, даже если никто не вызывал \t{signal}/\t{broadcast}.
		\item Поэтому важно проверять условие после пробуждения.
	\end{itemize}
\end{frame}

\begin{frame}[fragile]{Создание}
	Точно так же, как и мьютекс:
\begin{minted}{cpp}
pthread_cond_t cond;
pthread_cond_init(&cond);
// ...
pthread_cond_destroy(&cond);
\end{minted}
\end{frame}

\begin{frame}[fragile]{Оповещение}
\begin{minted}{cpp}
pthread_mutex_t m;
pthread_cond_t cond; // GUARDED_BY(m)
bool some_condition; // GUARDED_BY(m)
// ...
pthread_mutex_lock(&m);
// Следующие две строки в любом порядке.
some_condition = true;
pthread_cond_signal(&cond);
pthread_mutex_unlock(&m);
\end{minted}
\end{frame}

\begin{frame}[fragile]{Ожидание условия}
\begin{minted}{cpp}
pthread_mutex_t m;
pthread_cond_t cond; // GUARDED_BY(m)
bool some_condition; // GUARDED_BY(m)
// ...
pthread_mutex_lock(&m);
while (!some_condition) {
  // Атомарно снимает мьютекс и начинает ожидание
  pthread_cond_wait(&cond, &m);
  // После выхода из cond_wait мьютекс снова захвачен.
}
pthread_mutex_unlock(&m);
\end{minted}
\end{frame}
