\subsection{Знаковые числа}

\begin{frame}
	\tableofcontents[currentsection,currentsubsection]
\end{frame}

\begin{frame}
	Можно сказать, что в первом бите храним знак (прямой код).
	Тогда надо разбирать случаи в процессоре для всех операций и сравнения.
	Появляются $+0$ и $-0$, так что ещё и сравнение на равенство сильно менять.

	А можно сказать, что:
	\begin{align*}
		x &= x + 256 \mod 256 \\
		-56 &= -56 + 200 = 200 \mod 256
	\end{align*}
	Самое главное свойство числа $-x$ "--- это $(-x)+x=0$!.
\end{frame}

\begin{frame}
	\begin{center}
		\includegraphics[scale=0.3]{one-ring-to-rule.jpg}
	\end{center}
\end{frame}

\begin{frame}
	Алгебра говорит, что $-x$ "--- это \textit{обратный по сложению к $x$}.
	В кольцах он есть.

	Мы только поменяли, как мы интерпретируем числа, но не их битовую запись:
	\[
		-56 + 100 = -56 + 256 + 100 = 200 + 100 = \t{1100 1000} + \t{0110 0100} = \t{1 0010 1100} = \t{0010 1100} = 44
	\]
	Таким образом, сложение, вычитание, и даже умножение по-прежнему работают (спасибо алгебраистам, что доказали).
	Упражнение: проверить.

	С делением хуже:
	\[
		-10 / 5 = (256 - 10) / 5 = 246 / 5 = 49.5 = \t{???}
	\]
\end{frame}

\begin{frame}{Стандартная конвенция}
	Обычно разделяют отрезок ровно пополам: $[-128; 127]$.
	Тогда по самому старшему биту определяют знак: 1 "--- отрицательное, 0 "--- неотрицательное.
	Такая конвенция называется дополнительный код: отрицательное и положительно число в сумме дают нули или
	дополняют до степени двойки $2^8$.

	Надо разбирать случаи в сравнении чисел и в делении с остатком (поэтому они в ассемблере появляются знаковые/беззнаковые).

	Операция смены знака: инвертировать все биты и добавить единицу, так как инвертация бит "--- вычитание из \t{1111 1111} (255).

	Упражнение: как представлены -1, -128, 127, 128?
	128 никак не представлено, есть некоторая асимметрия.
	Что будет, если мы возьмём $-(-128)$?
\end{frame}
