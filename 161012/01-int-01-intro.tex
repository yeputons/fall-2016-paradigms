\section{Целые числа}
\subsection{Физическая часть}

\begin{frame}
	\tableofcontents[currentsection,currentsubsection]
\end{frame}

\begin{frame}
	Упрощённое представление:
	\begin{enumerate}
		\item Два уровня напряжения распознавать проще, чем три.
		\item Но три тоже было, не прижилось.
		\item Всё построено на основе бинарных функций <<И>>, <<ИЛИ>> и остальных (\textit{гейты})
		\item Чем меньше гейтов "--- тем быстрее работает, тем меньше схема.
	\end{enumerate}
\end{frame}

\subsection{Типы данных}
\begin{frame}
	Чему соответствует бинарная запись в таблице ниже?
	\begin{center}
		\pause
		\begin{tabular}{|c|c|}
			\hline
			\t{0001 0110} & \pause 22 \\\hline\noalign{\pause}
			\t{1000 0010} & \pause 130 \\\hline\noalign{\pause}
			\t{1000 0010} & \pause -126 \\\hline\noalign{\pause}
			\t{0011 0000} & \pause 48 \\\hline\noalign{\pause}
			\t{0011 0000} & \pause '0' \\\hline\noalign{\pause}
			\t{1100 0011} & \pause \t{0xC3} \\\hline\noalign{\pause}
			\t{1100 0011} & \pause \t{ret} \\\hline
		\end{tabular}
		\pause
	\end{center}
	Мораль: битовое представление ничего не говорит, если мы не договорились о том,
	как его интерпретировать (<<тип>>).

	Более того, представлений у одной и той же сущности может быть в
	некотором смысле много (\t{0xC3}, 195, \t{ret}).
\end{frame}
