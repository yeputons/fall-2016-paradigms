\subsection{IEEE 754}
\begin{frame}
	\tableofcontents[currentsection,currentsubsection]
\end{frame}

\begin{frame}
	Есть стандарт IEEE 754 для хранения и обработки вещественных чисел с плавающей запятой, который используется почти везде.
	Там определено несколько типов данных с разными размерами мантисс и экспонент.

	IEEE 754 сложнее предыдущих слайдов, чтобы выполнялись всякие полезные свойства и операции всегда были определены.
	Например, есть $\pm \infty$ и \t{NaN} (Not a Number, получается при делении на ноль).
	Разберём \textit{single-precision} (в C++/Java "--- \t{float}):

	\begin{center}
		\begin{tabular}{|r|c|c|c|}
			\hline
			Биты & 31 & 30-23 & 22-0 \\\hline
			Тип & знак ($s$) & экспонента ($e$) & мантисса ($m$) \\\hline
		\end{tabular}
	\end{center}
	Если $s=0$, то число положительное, иначе отрицательное.
	Экспонента хранится не в дополнительном коде, а со сдвигом на 127 (т.е. $\t{0000 0000}=-127$, а $\t{0111 1111}=0$).
	Предполагается, что экспонента подобрана так, чтобы перед запятой был ровно один знак "--- единица.
	А мантисса тогда хранит все остальные.

	Итоговая формула:
	\[
		x = (-1)^s \cdot 2^{e-127} \cdot (1 + m \cdot 2^{-23})
	\]
\end{frame}

\begin{frame}{Пример}
	\[
		x = -13.75_{10} = \frac{-220}{16} = -1101.110_2
	\]
	Подбираем экспоненту так, чтобы получилась единица:
	\[
		x = -1.\underbrace{101110_2}_{m'=92_{10}} \cdot 2^3 \quad e=3
	\]
	Так как в $m$ предполагается 23 значащих знака (а у нас только 6), надо дописать \textit{справа} нулей.
	Итого:
	\begin{center}
		\begin{tabular}{|r|c|c|c|}
			\hline
			Биты & 31 & 30-23 & 22-0 \\\hline
			Тип & \t{1} & $\underbrace{\t{100 0001 0}}_{e+128}$ & $\underbrace{\textit{101 110}0 0000 0000 0000 0000}_{m}$ \\\hline
		\end{tabular}
	\end{center}	
\end{frame}
