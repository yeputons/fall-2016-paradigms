\subsection{IEEE 754}
\begin{frame}
	\tableofcontents[currentsection,currentsubsection]
\end{frame}

\begin{frame}
	Есть стандарт IEEE 754 для хранения и обработки вещественных чисел с плавающей запятой, который используется почти везде.
	Там определено несколько типов данных с разными размерами мантисс и экспонент.

	IEEE 754 сложнее предыдущих слайдов, чтобы выполнялись всякие полезные свойства и операции всегда были определены.
	Например, есть $\pm \infty$ и \t{NaN} (Not a Number, получается при делении на ноль).
	Разберём \textit{single-precision} (в C++/Java "--- \t{float}):

	\begin{center}
		\begin{tabular}{|r|c|c|c|}
			\hline
			Биты & 31 & 30-23 & 22-0 \\\hline
			& знак ($s$) & экспонента ($e$) & мантисса ($m$) \\\hline
		\end{tabular}
	\end{center}
	Если $s=0$, то число положительное, иначе отрицательное.
	Экспонента хранится не в дополнительном коде, а со сдвигом на 127 (т.е. $\t{0000 0000}=-127$, а $\t{0111 1111}=0$)
	и может принимать значения от $-127$ до $128$ (\textbf{не} от $-128$ до $127$).
	Предполагается, что экспонента подобрана так, чтобы перед запятой был ровно один знак "--- единица.
	А мантисса тогда хранит все остальные.

	Итоговая формула (для \textit{нормализованных} чисел):
	\[
		x = (-1)^s \cdot 2^{e-127} \cdot (1 + m \cdot 2^{-23})
	\]
\end{frame}

\begin{frame}{Пример}
	\[
		x = -13.75_{10} = \frac{-220}{16} = -1101.110_2
	\]
	Подбираем экспоненту так, чтобы получилась единица:
	\[
		x = -1.\underbrace{101110_2}_{m'=92_{10}} \cdot 2^3 \quad e=3
	\]
	Так как в $m$ предполагается 23 значащих знака (а у нас только 6), надо дописать \textit{справа} нулей.
	Итого:
	\begin{center}
		\begin{tabular}{|r|c|c|c|}
			\hline
			Биты & 31 & 30-23 & 22-0 \\\hline
			& \t{1} & $\underbrace{\t{100 0001 0}}_{e+128}$ & $\underbrace{\textit{101 110}0 0000 0000 0000 0000}_{m}$ \\\hline
		\end{tabular}
	\end{center}	
\end{frame}

\begin{frame}{Денормализованные числа}
	Какое наименьшее нормализованное число можно представить в таком формате?
	Очевидно, при минимальной экспоненте (-127) и мантиссе.
	Тогда самые маленькие числа таковы:
	\begin{align*}
		x &= 2^{-127} \cdot 1 \\
		y &= 2^{-127} \cdot (1 + 1 \cdot 2^{-23}) \\
		z &= 2^{-127} \cdot (1 + 2 \cdot 2^{-23}) \\
		& \vdots \\
	\end{align*}
	Посчитаем что-нибудь:
	\begin{align*}
		x + y &= 2^{-127} \cdot (1 + 1 + 1 \cdot 2^{-23}) = \\
		      &= 2^{-127} \cdot (2 + 1 \cdot 2^{-23}) = \\
		      &= 2^{-126} \cdot (1 + 1 \cdot 2^{-24}) \approx \\
		      &\approx 2^{-126} \cdot 1 \\
		y - x &= 2^{-127} \cdot (1 + 1 \cdot 2^{-23} - 1) = \\
		      &= 2^{-127} \cdot 1 \cdot 2^{-23} = 2^{-150}
	\end{align*}
	Округлять к $2^{-127}$ странно (и тогда бы получили, что $x-y=x$, но $x\neq x + y$).
	Округлять к нулю тоже странно: $x \neq y$, но $x - y = 0$.

	Так что вблизи нуля добавили \textit{денормализованные числа}, чтобы повысить точность и избежать таких проблем (ведь компилятор-то хочет оптимизировать).
	Тогда эта проблема исчезает.
\end{frame}

\begin{frame}{Формат}
	Скажем, что если экспонента состоит из нулей, то у нас денормализованное число:
	\begin{center}
		\begin{tabular}{|r|c|c|c|}
			\hline
			Биты & 31 & 30-23 & 22-0 \\\hline
			Тип & знак ($s$) & нули & мантисса $m$ \\\hline
		\end{tabular}
	\end{center}
	Тут мы уже считаем, что мантисса записана целиком, с учётом старшей единицы.
	Формула:
	\[
		x = (-1)^s \cdot 2^{-12\textbf{6}} \cdot m \cdot 2^{-23}
	\]
	Тогда денормализованные числа лежат в диапазоне:
	\[
		2^{-126} \cdot 2^{-23} \le x \le 2^{-126} \cdot (2^{23}-1) \cdot2^{-23}
	\]
	А нормализованные (так как экспонента из нулей, т.е. $-127$, убита):
	\[
		2^{-126} \cdot 1 \le x
	\]
\end{frame}

\begin{frame}
	\begin{center}
		\includegraphics[scale=0.75]{what-are-you-doing.jpg}
	\end{center}
\end{frame}

\begin{frame}{Все особенности IEEE-754}
	Нормализованных чисел с экспонентой $-127$ не существует "--- это будут \textit{денормализованные} числа (см. следующий слайд).
	Есть два нуля "--- по одному с каждым знаком (попробуйте их сравнить в Python, а потом на C++ на них поделить).
	Сделано для некой внутренней консистентности.

	Нормализованных чисел с экспонентой $128$ не существует "--- это будут либо $+\infty$, либо NaN (not a number).
	Если мантисса из нулей "--- то бесконечность (в зависимости от знака), если есть хоть один не-ноль "--- NaN.
	\begin{center}
		\begin{tabular}{|r|c|c|c|c|}
			\hline
			Биты & 31 & 30-23 & 22-0 & Значение \\\hline
			& \t{0} & \t{111 1111 1} & \t{000 0000 0000 0000 0000 0000} & $+\infty $\\\hline
			& \t{1} & \t{111 1111 1} & \t{000 0000 0000 0000 0000 0000} & $-\infty $\\\hline
			& \t{0} & \t{111 1111 1} & \t{000 1000 0110 0000 0000 0000} & \t{NaN} \\\hline
			& \t{0} & \t{000 0000 0} & \t{000 0000 0000 0000 0000 0000} & $+0$ \\\hline
			& \t{0} & \t{000 0000 0} & \t{000 0000 0000 0000 0000 0000} & $-0$ \\\hline
		\end{tabular}
	\end{center}	
\end{frame}
