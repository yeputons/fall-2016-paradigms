\subsection{Порядок байт}

\begin{frame}
	\tableofcontents[currentsection,currentsubsection]
\end{frame}

\begin{frame}
	Напоминание: порядок бит в байте мы из программы никак не определим, на картинке "--- слева старшие, справа младшие.
	Порядок байт в памяти "--- слева меньшие адреса, справа большие.

	Игра:
	\begin{center}
		\pause
		\begin{tabular}{|c|c|c|c|c|c|c|}
			\hline
			\multicolumn{5}{|c|}{Адрес} & \\\hline
			0 & 1 & 2 & 3 & 4 & Значение \\\hline
			\dots & \dots & \t{0000 0001} & \t{0000 0011} & \dots & \pause 259 \\\hline\noalign{\pause}
			\dots & \dots & \t{0000 0001} & \t{0000 0011} & \dots & \pause 769 \\\hline
		\end{tabular}
		\pause
	\end{center}

	Да как договоримся, так и читать.
	Байты бывают младшие и старшие.
	Договариваются по-разному на разных процессорах и в разных протоколах.
	Свойство <<порядок байт>> называется endianness.
\end{frame}

\begin{frame}

	\begin{center}
		\includegraphics[scale=0.3]{eggs.jpg}
	\end{center}

	Есть два клана: little-endian (остроконечники) и big-endian (тупоконечники).
	По-русски всегда используют английские термины.
\end{frame}

\begin{frame}{Big-endian}
	Используется в низкоуровневых сетевых протоколах (TCP) и процессорах Atmel AVR (ATmega и прочие).

	Младший байт имеет больший адрес.
	Тогда запись совпадает с <<естественной>>:
	\begin{center}
		\begin{tabular}{|c|c|c|c|c|c|c|}
			\hline
			\multicolumn{5}{|c|}{Адрес} & \\\hline
			0 & 1 & 2 & 3 & 4 & Значение \\\hline
			\dots & \dots & \t{0000 0001} & \t{0000 0011} & \dots & $\t{0000 00\textbf{01} 0000 00\textbf{11}}_2=769_{10}$ \\\hline
		\end{tabular}
	\end{center}
\end{frame}

\begin{frame}{Little-endian}
	Используется в x86.

	Младший байт имеет меньший адрес.
	Читается хуже:
	\begin{center}
		\begin{tabular}{|c|c|c|c|c|c|c|}
			\hline
			\multicolumn{5}{|c|}{Адрес} & \\\hline
			0 & 1 & 2 & 3 & 4 & Значение \\\hline
			\dots & \dots & \t{0000 0001} & \t{0000 0011} & \dots & $\t{0000 00\textbf{11} 0000 00\textbf{01}}_2=259_{10}$ \\\hline
		\end{tabular}
	\end{center}
	Если только мы не Intel и не пишем к этому документацию:
	\begin{center}
		\begin{tabular}{|c|c|c|c|c|c|c|}
			\hline
			\multicolumn{5}{|c|}{Адрес} & \\\hline
			4 & 3 & 2 & 1 & 0 & Значение \\\hline
			\dots & \dots & \t{1100 0000} & \t{1000 0000} & \dots & $\t{0000 0011 0000 0001}_2=259_{10}$ \\\hline
		\end{tabular}
	\end{center}
\end{frame}
