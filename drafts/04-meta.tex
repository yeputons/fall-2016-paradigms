\section{Метаклассы}

\begin{frame}
	\tableofcontents[currentsection,currentsubsection]
\end{frame}

\begin{frame}[t,fragile]{Идея}
\begin{minted}{python}
class A:
    pass
a = A()
print(type(a))  # <class '__main__.A'>
print(type(A))  # type
\end{minted}
	\begin{itemize}
		\item В Python всё динамическое, и почти всё в некотором смысле "--- объект (имеет методы и свойства, которые можно менять).
		\item Так зачем разделять <<классы>> и <<объекты>> в языке, если они ничем по поведению не отличаются, только по использованию?
		\only<2->{
		\item Классы "--- это тоже объекты класса \t{type}.
		\item \t{type} "--- это тоже класс. И тоже объект класса \only<2>{...}\only<3->{\t{type}. Этакая рекурсия.
		\item Обычно метаклассы в программах писать и использовать не требуется.
		}}
	\end{itemize}
\end{frame}

\begin{frame}{Замечание}
	В других языках я ничего похожего не видел, кроме, пожалуй JavaScript (там свои тараканы).

	Обычно есть чёткое разделение на <<класс>> (статическое описание структуры объектов) и <<объект>>.
\end{frame}

\begin{frame}[fragile]{Зачем}
	Используется, когда хочется изменить стандартное поведение слова \t{class}:
	\begin{itemize}
		\item Автоматически сгенерировать новые методы.
		\item Проверить, что в классе определены все нужные методы.
		\item Зарегистрировать класс где-нибудь при создании (например, если мы хотим знать все классы, которые у нас есть).
		\item Автоматически обернуть все методы в декоратор (для логов, для замеров времени).
	\end{itemize}
\end{frame}

\begin{frame}[fragile]{Применение}
	Обычно метаклассы незаметны; авторы библиотеки просто говорят, как надо писать код:
\begin{minted}{python}
from django.db import models
class Person(models.Model):
    first_name = models.CharField(max_length=30)
    last_name = models.CharField(max_length=30)
\end{minted}
	За этими тремя строками скрывается:
	\begin{itemize}
		\item Автоматическая генерация таблицы в базе данных.
		\item Создание страницы сайта для редактирования.
		\item Создание конструктора с параметрами \t{first\_name} и \t{last\_name}.
		\item Создание методов для загрузки и сохранения объектов в БД.
		\item Создание методов для проверки корректности данных.
		\item Создание методов \t{\_\_eq\_\_}, \t{\_\_str\_\_} и других.
	\end{itemize}
\end{frame}

\begin{frame}[fragile]{Пример}
\begin{minted}{python}
class NamedTupleMeta(type):
    # Тут используют метод __new__ вместо __init__.
    def __new__(cls, name, bases, dct):  # cls, не self.
        print(cls, name, bases, dct)
        attrs = dct["attrs"]
        def init(self, **kwargs):
            for key in kwargs:
                assert key in attrs, "Unknown key: " + key
                setattr(self, key, kwargs[key])
        dct["__init__"] = init
        del dct["attrs"]
        return super().__new__(cls, name, bases, dct)

class Person(metaclass=NamedTupleMeta):
    attrs = ["first_name", "last_name"]
\end{minted}
\end{frame}
