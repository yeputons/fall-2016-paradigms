\documentclass[utf8,xcolor=table]{beamer}

\usepackage[T2A]{fontenc}
\usepackage[utf8]{inputenc}
\usepackage[english,russian]{babel}
%\usepackage{tikz-uml}
\usepackage{minted}

\mode<presentation>{
	\usetheme{CambridgeUS}
}

\renewcommand{\t}[1]{\ifmmode{\mathtt{#1}}\else{\texttt{#1}}\fi}

\title{Матрицы, numpy и классы}
\author{Егор Суворов}
\institute[СПб АУ]{Курс <<Парадигмы и языки программирования>>, подгруппа 3}
\date[21.09.2016]{Среда, 21 сентября 2016 года}

\setlength{\arrayrulewidth}{1pt}

\input{../private/contacts}

\begin{document}

\begin{frame}
\titlepage
\end{frame}

\begin{frame}[t]{Организационное}
	\begin{itemize}
	\item Я студент третьего курса бакалавриата СПб АУ (программист).
	\item Имена: <<извините, пожалуйста>>, <<Егор>>, <<Егор Фёдорович>>.
	\item
		\begin{tabular}{rl}
			E-mail: & \href{mailto:\EgorEmail?subject=[parad]}{\t{\EgorEmail}} \\
			Тема e-mail: & \t{[parad]...} \\
			ВК: & \href{https://vk.com/\EgorVk}{\t{\EgorVk}} \\
			Telegram: & \href{https://telegram.me/\EgorTelegram}{\t{\EgorTelegram}}
		\end{tabular}
	\item
		Домашние задания общие с остальными подгруппами.
	\item
		Решения надо присылать мне.
		Если уже прислали кому-то ещё "--- перенаправлять не надо.
	\item
		Вопросы по текущей теме или смежным можно задавать в ходе рассказа на занятии.
	\item
		Вопросы по остальным темам и предметам лучше в оффлайне.
	\item
		А хотите Office Hours с 10 до 11?
	\item
		Слайды "--- на SEWiki.
	\end{itemize}
\end{frame}

\begin{frame}[t]{Зачёт и проверка}
	\begin{itemize}
		\item Надо набрать хотя бы половину баллов от максимума.
		\item Надо набрать строго положительные баллы по каждой теме.
	\end{itemize}
\end{frame}

\begin{frame}[t]{Кто такие матрицы}
	\framebox[\linewidth]{
		\textit{Матрица} "--- это квадратная\footnote{\only<2->{прямоугольная}} табличка\footnote{\only<3->{алгебраический объект}} с чиселками\footnote{\onslide<4->{необязательно числа}}.
	}

	\only<5->{
		Полезны при изучении:
		\begin{itemize}
			\item Почти чего угодно со словом <<линейное>>:
				\begin{itemize}
					\item Систем линейных уравнений.
					\item Линейных преобразований (плоскости, пространства...).
					\item Линейных рекуррент ($F_n=F_{n-1}+F_{n-2}$).
				\end{itemize}
			\item Графов (матрица смежности/инцидентности).
			\item Машинного обучения (привет от линейной алгебры).
		\end{itemize}
	}
\end{frame}

\begin{frame}[t]{Операции с матрицами}
	Запись матрицы $A$ размера $2 \times 3$:
	\begin{align*}
		A &= \begin{pmatrix}
			A_{1,1} & A_{1,2} & A_{1,3} \\
			A_{2,1} & A_{2,2} & A_{2,3} \\
		\end{pmatrix}
	\end{align*}

	Сложение матриц одного размера:
	\[
		\begin{pmatrix}
			A_{1,1} & A_{1,2} \\
			A_{2,1} & A_{2,2} \\
			0 & A_{3,2} \\
		\end{pmatrix}
		+
		\begin{pmatrix}
			B_{1,1} & B_{1,2} \\
			B_{2,1} & 0 \\
			B_{3,1} & B_{3,2} \\
		\end{pmatrix}
		=
		\begin{pmatrix}
			A_{1,1}+B_{1,1} & A_{1,2}+B_{1,2} \\
			A_{2,1}+B_{2,1} & A_{2,2} \\
			B_{3,1} & A_{3,2}+B_{3,2} \\
		\end{pmatrix}
	\]

	Уможение\footnote{интуитивно можно думать <<композиция отображений>>} матрицы с $k$ столбцами на матрицу с $k$ строками:
	\[
		\begin{pmatrix}
			A_{1,1} & A_{1,2} & A_{1,3} \\
			A_{2,1} & A_{2,2} & A_{2,3} \\
		\end{pmatrix}
		\cdot
		\begin{pmatrix}
			B_{1,1} \\
			B_{2,1} \\
			B_{3,1} \\
		\end{pmatrix}
		=
		\begin{pmatrix}
			A_{1,1} \cdot B_{1,1} + A_{1,2} \cdot B_{2,1} + A_{1,3} \cdot B_{3,1} \\
			A_{2,1} \cdot B_{1,1} + A_{2,2} \cdot B_{2,1} + A_{2,3} \cdot B_{3,1} \\
		\end{pmatrix}
	\]
\end{frame}

\begin{frame}[t]{Ага}
	\begin{center}
		\includegraphics[width=0.5\linewidth]{multiplication-meme.jpg}
	\end{center}
\end{frame}

\begin{frame}[t]{Wait But Why}
	Например, если матрица описывает преобразование координат:
	\begin{align*}
		\begin{pmatrix} x_2 \\ y_2 \end{pmatrix} &=
			\begin{pmatrix}ax_1 + by_1 \\ cx_1 + dy_1\end{pmatrix} =
			\underbrace{\begin{pmatrix}a & b \\ c & d\end{pmatrix}}_{A} \cdot \begin{pmatrix}x_1 \\ y_1\end{pmatrix} \\
		\begin{pmatrix} x_3 \\ y_3 \end{pmatrix} &=
			\begin{pmatrix}a'x_2 + b'y_2 \\ c'x_2 + dy_2\end{pmatrix} =
			\underbrace{\begin{pmatrix}a' & b' \\ c' & d'\end{pmatrix}}_{A'} \cdot \begin{pmatrix}x_2 \\ y_2\end{pmatrix}
			\only<1>{\phantom{=A' \cdot \begin{pmatrix}x_2 \\ y_2\end{pmatrix} = A' \cdot \left(A \cdot \begin{pmatrix} x_1 \\ y_1\end{pmatrix}\right)}}
			\only<2->{=A' \cdot \begin{pmatrix}x_2 \\ y_2\end{pmatrix} = A' \cdot \left(A \cdot \begin{pmatrix} x_1 \\ y_1\end{pmatrix}\right)}
			\\
		\begin{pmatrix} x_3 \\ y_3 \end{pmatrix} &=
			\begin{pmatrix}a'(ax_1+by_1) + b'(cx_1+dy_1) \\ c'(ax_1+by_1) + d'(cx_1+dy_1)\end{pmatrix} = \\
		&=	\begin{pmatrix}(a'a+b'c)x_1+(a'b+b'd)y_1 \\ (c'a+d'c)x_1+(c'b+d'd)y_1 \end{pmatrix} =
			(A'\cdot A)\cdot\begin{pmatrix}x_1 \\ y_1\end{pmatrix}
	\end{align*}
\end{frame}

\begin{frame}[t,fragile]{Умножение матриц-1}
\begin{minted}{python}
a=[[1,2,3],
   [4,5,6]]
b=[[5,4],
   [3,2],
   [1,0]]
c=[[0, 0], [0,0]]
for i in range(2):
  for j in range(2):
    c[i][j] = sum(a[i][k] * b[k][j] for k in range(3))
print(c)  # [[14, 8],
          #  [41, 26]]
\end{minted}
	Получаем $O(n \cdot m \cdot k)$.
	Если матрицы квадратные "--- куб.

	Python медленный, так что у меня работает 0.65 секунд на матрицах $150 \times 150$.
\end{frame}

\begin{frame}[t]{Умножение матриц-2}
	1987 год "--- алгоритм Копперсмита-Винограда, $O(n^{2,375477})$.

	2010--2014 года "--- серия улучшений, аж до $O(n^{2,3728639})$.

	Практика "--- алгоритм Штрассена, $O(n^{2,807355})$.
	Вам надо будет его реализовать.
\end{frame}

\begin{frame}[t]{Алгоритм Штрассена}
	Разделяй-и-властвуй (как алгоритм Карацубы):
	\begin{enumerate}
		\item
			Разделили каждую матрицу на четыре одинаковых.
		\item
			???
		\item
			Profit
	\end{enumerate}

	Цель "--- чтобы на втором шаге потребовалось строго меньше восьми умножений матриц:
	\begin{gather*}
		\begin{pmatrix} A_{1,1} & A_{1,2} \\ A_{2,1} & A_{2,2} \end{pmatrix}
		\cdot
		\begin{pmatrix} B_{1,1} & B_{1,2} \\ B_{2,1} & B_{2,2} \end{pmatrix}
		= \\
		=
		\begin{pmatrix}
			A_{1,1}\cdot B_{1,1}+A_{1,2}\cdot B_{2,1} & A_{1,1} \cdot B_{1,2} + A_{1,2} \cdot B_{2,2} \\
			A_{2,1}\cdot B_{1,1}+A_{2,2}\cdot B_{2,1} & A_{2,1} \cdot B_{1,2} + A_{2,2} \cdot B_{2,2} \\
		\end{pmatrix}
	\end{gather*}
	Волшебные формулы смотрим в Википедии.
\end{frame}

\begin{frame}[t,fragile]{Велосипед уже изобретён}
	NumPy "--- библиотека для работы с многомерными массивами, в том числе с матрицами.
	\begin{itemize}
		\item Написано на C, незаметно <<встраивается>> в Python.
		\item Может даже использовать GPU, если хочет.
		\item Массивы гарантированно хранятся более эффективно.
		\item Ускоряются только встроенные в библиотеку операции (не циклы).
	\end{itemize}
\begin{minted}{python}
import numpy as np
a = np.array([[1,2,3],[4,5,6]])
b = np.array([[7,8,9],[10,11,12]])
print(a.shape)
print(a+b)
b[1][1] += 100
print(a.dot(b.transpose()))
print(b[:, 1:])
print(b[np.array([0, 0])])
\end{minted}
\end{frame}

\begin{frame}[t,fragile]{Форма и копии}
	\begin{itemize}
		\item
			Можно считать, что все элементы массива расположены в памяти в лексикографическом порядке друг за другом.
		\item
			Форма массива "--- это лишь числа, которые легко меняются:
			\begin{minted}{python}
print(np.arange(24))
print(np.arange(24).reshape(3, 8))
print(np.arange(24).reshape(3, 8).reshape(2, 3, 4))
			\end{minted}
		\item
			Все операции по умолчанию поэлементные (сложение, \t{np.exp}...).
		\item
			Конструкция \t{a = b} ничего не копирует (только ссылку).
		\item
			Можно получить отдельный объект на том же массиве (view): \t{b~=~a.view()}, \t{b~=~a[...]} (\t{...} "--- это именно многоточие).
		\item
			Можно скопировать целиком (долго): \t{b = a.copy()}.
		\item
			Можно конкатенировать массивы: \t{vstack}, \t{hstack}.
	\end{itemize}
\end{frame}

\begin{frame}[t,fragile]{Распространение поэлементных операций}
\begin{minted}{python}
import numpy as np
a = np.arange(10).reshape(2, 5)
print(a * 100)
print(a + [5, 4, 3, 2, 1])  
print(a + [[100], [1000]])
print(a.reshape(1, 10) * a.reshape(10, 1))
\end{minted}
	\begin{itemize}
		\item К размерности дописываются единицы: $(3) \to (1, 3) \to (1, 1, 3) \to \dots$
		\item Если какая-то размерность равна единице, то её можно расширить до максимума: $(1, 10) \to (10, 10)$.
	\end{itemize}
\end{frame}

\begin{frame}[t,fragile]{Матрицы}
	Матрица "--- это двухмерный массив.
\begin{minted}{python}
import numpy as np
a = np.arange(4).reshape(2, 2)
print(a * a)  # wtf
\end{minted}

	\only<2->{
	Потому что все операции поэлементные. Используйте \t{a.dot(a)} или \t{np.dot(a, a)}.

	Альтернативно можно использовать \t{np.matrix} вместо \t{np.array}, но это не рекомендуется:
	\begin{itemize}
		\item Если использовать оба, то будет путаница в операциях (типизация-то динамическая).
		\item Некоторые функции сторонних библиотек могут вернуть \t{array}, даже если получили на вход \t{matrix}.
		\item Поэлементные операции не работают (хотя умножать на число можно).
		\item \t{*} и \t{/} ведут себя по-разному.
	\end{itemize}
	}
\end{frame}

\begin{frame}[t,fragile]{Задание в классе №1(а)}
	Обязательное:
	\begin{enumerate}
		\item Установите NumPy (в составе набора SciPy): \href{http://www.scipy.org/install.html}{scipy.org/install.html}.
			\begin{itemize}
				\item Для Ubuntu всё лежит в репозиториях.
				\item Для Mac надо поставить специальную сборку Python, в которую всё включено.
				\item Для Windows надо сделать один из пунктов (на выбор):
					\begin{itemize}
						\item Выполнить в консоли (запускать <<от имени администратора>>):
							\t{pip~install~numpy}. Скачает десяток мегабайт и всё поставит.
						\item Поставить специальную сборку Python вроде WinPython (см. ссылку).
					\end{itemize}
			\end{itemize}
		\item Откройте руководство: \href{http://docs.scipy.org/doc/numpy-dev/user/quickstart.html}{docs.scipy.org/doc/numpy-dev/user/quickstart.html}.
		\item Перемножьте две матрицы при помощи NumPy и покажите код.
	\end{enumerate}
\end{frame}

\begin{frame}[t,fragile]{Задание в классе №1(б)}
	Дополнительное:
	\begin{enumerate}
		\setcounter{enumi}{3}
		\item Перемножьте две такие же матрицы при помощи циклов \t{for} без NumPy.
		\item Уменьшите количество циклов \t{for} до двух.
		\item Замерьте время работы двух функций при разных $n$ при помощи модуля \t{timeit}: \t{print(timeit.Timer(some\_func).timeit(1))}.
		\item Поделитесь разницей в скорости с друзьями.
	\end{enumerate}
\end{frame}

\begin{frame}[t,fragile]{Набираем классы}
Если логическая сущность состоит из нескольких кусочков данных, они все лежат в одном объекте:
\begin{minted}{python}
class BinaryExpression:
    def __init__(self, left, op, right):
        self.left = left
        self.op = op
        self.right = right

    def validate(self):
        self.left.validate()
        assert self.op == '+'
        self.right.validate()

expr = BinaryExpression(None, '+', None)
expr.validate()  # fails
\end{minted}
\end{frame}

\begin{frame}[t,fragile]{Классы по-змеиному}
	\begin{itemize}
		\item Первый параметр у методов объекта (передаётся неявно, в отличие от Java/C++) "--- это объект, над которым совершается операция.
		\item Обычно первый параметр называют \t{self}
		\item \t{\_\_init\_\_} "--- конструктор.
        \item Можно переопределить метод \t{\_\_str\_\_}, чтобы работал \t{print(expr)}.
        \item Также бывает \t{\_\_cmp\_\_}, \t{\_\_eq\_\_} и другие <<magic methods>>
        \item Всё публичное; приватное принято просто не вызывать. Можно начать метод с \t{\_}, чтобы подчеркнуть приватность.
        \item Типизация утиная, никакого строгого <<интерфейса>> не задать.
	\end{itemize}
\end{frame}

\begin{frame}[t,fragile]{Стандартные проблемы}
\begin{minted}{python}
class Foo:
    x=[]
foo = Foo()
bar = Foo()
print(foo.x, bar.x)
foo.x.append(1)
print(foo.x, bar.x)
\end{minted}

	Как с аргументами по умолчанию в функциях:
\begin{minted}{python}
def foo(x=[]):
    x.append(1)
    print(x)
foo()
foo()
\end{minted}

	Поле \t{x} было одно на все экземпляры \t{Foo}.
\end{frame}

\begin{frame}[t]{Задание в классе №2}
	Напишите несколько программ, которые читают со стандартного входа строчки по одной и после каждого чтения выводят текущую строчку, если:
	\begin{enumerate}
		\item Она имела чётный номер.
		\item Она была длины 3.
		\item Она представляла собой целое число.
		\item Она имела чётный номер среди строк длины 3, являющихся целыми числами.
	\end{enumerate}
	Запрещается хранить предыдущие введённые строки.

	После этого сохраните код в отдельный файл и выполните следующее задание.
\end{frame}

\begin{frame}[t]{Задание в классе №3}
	\begin{itemize}
		\setcounter{enumi}{4}
		\item
			Создайте и используйте класс <<\t{Printer}>> с методом \t{consume(str)}, который печатает строку \t{str}.
		\item
			Создайте и используйте класс <<\t{FilterLengthIs3}>> с конструктором \t{\_\_init\_\_(consumer)} и методом \t{consume(str)}.
			При вызове \t{consumer(str)} на строке длины три должен быть вызван метод \t{consumer.consume(str)}.
		\item
			Создайте и используйте аналогичный класс <<\t{FilterIsNumber}>>.
		\item
			Создайте и используйте ангалогичный класс <<\t{DropOdd}>>.
		\item
			Зацените чёткое разделение ответственности между написанными классами по сравнению с кодом без классов.
	\end{itemize}
\end{frame}

\begin{frame}[t]{Работаем и спрашиваем!}
\end{frame}

\end{document}
