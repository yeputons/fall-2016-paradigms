\subsection{Реляционные СУБД и простой SQL}

\begin{frame}{Реляционные СУБД на практике}
	\begin{itemize}
		\item СУБД хранит одну или несколько независимых БД (баз данных).
		\item Каждая БД "--- это набор таблиц, которые содержат данные.
		\item Таблица имеет фиксированный набор столбцов с названиями и типами.
		\item В таблице лежит неупорядоченный набор строк с данными.
		\item Обычно запросы к реляционным СУБД формулируются на декларативном языке SQL
			(Structured Query Language).
	\end{itemize}
\end{frame}

\begin{frame}{Реляционная алгебра}
	Математическая модель происходящего в реляционных СУБД:
	\begin{itemize}
		\item Таблица называется \textit{отношением} (relation, отсюда relational database).
		\item Есть операции над таблицами (образующие алгебру).
			Например, <<выбрать какие-то строчки из таблицы>>.
		\item Обычно СУБД поддерживают гораздо более крутые и странные операции,
			чем в реляционной алгебре.
		\item Больше слова <<реляционная алгебра>> вам наверняка не пригодятся.
	\end{itemize}
\end{frame}
