\subsection{Использование СУБД}

\begin{frame}
	\tableofcontents[currentsection,currentsubsection]
\end{frame}

\begin{frame}[t]{Где и зачем}
	СУБД используются практически везде:
	\begin{itemize}
		\only<1>{
		\item
			Если данных или клиентов (которые запрашивают/меняют данные) будет очень много,
			то нам не надо изобретать велосипед и писать своё масштабируемое хранилище:
			\begin{enumerate}
				\item Обычно одна СУБД обслуживает сразу несколько приложений.
				\item Можно создавать разных пользователей с разными правами.
				\item Можно прозрачно для приложений делать бэкапы или хранить данные на десяти серверах.
			\end{enumerate}
		}
		\only<2>{
		\item
			Если мы просто пишем приложение с какой-то нетривиальной схемой:
			\begin{enumerate}
				\item Очень чётко отделяются данные от их обработки.
				\item SQL все знают (в отличие от логики программы), легко делать запросы к БД, зная только схему, и не зная ничего про приложение.
				\item SQL мощнее и читается лучше циклов for и list comprehension, которые ещё и не во всех языках есть.
				\item Не надо думать про хранение данных.
			\end{enumerate}
		}
		\only<3>{
		\item
			Если мы data scientist и/или хотим активно проверять гипотезы и много/просто работать с данными:
			\begin{enumerate}
				\item Удобно, когда все данные лежат в БД с известным интерфейсом (SQL).
				\item Не надо писать никакой код и ни с чем интегрироваться, чтобы выполнить запрос.
				\item Не получится набагать в коде в обработке крайних случаев\footnote{Даже в SQL можно посадить сложный баг}
			\end{enumerate}
		}
	\end{itemize}
\end{frame}

\begin{frame}{В чём минусы}
	\begin{itemize}
		\item
			Мы отдаём контроль за скоростью выполнения и потреблением памяти в руки СУБД
			(как и при любой абстракции).
			Это обычно приемлимый компромисс.
		\item
			Приложение сложнее запустить: нужно настроить СУБД, что обычно занимает несколько шагов.
			В нестадартных ситуациях "--- больше.
		\item
			Иногда приложение требует слишком хитрую настройку СУБД (например, для корректной работы
			с не-латиницей и датами).
		\item
			Многие инструменты заточены под промышленные решения и имеют слишком много рычажков и кнопок
			для простых целей.
	\end{itemize}
\end{frame}

\begin{frame}{Встраиваемые СУБД}
	\begin{itemize}
		\item Самая известная встраиваемая СУБД "--- sqlite.
		\item Предназначена не для сетевого доступа, а для использования в рамках одной конкретной программы.
		\item Её можно просто вкомпилировать в своё приложение, не требуется никакой настройки.
		\item sqlite хранит каждую БД в отдельном файле определённого формата (последний "--- sqlite3).
		\item Формат sqlite3 один на все приложения, можно даже залезть в чужие БД и посмотреть.
		\item Занимает мало места в скомпилированном приложении.
		\item
			Используется \href{http://www.sqlite.org/famous.html}{во многих приложениях}:
			под Android, в Firefox, в Chrome, в клиенте Dropbox\footnote{ищите файлы \t{.db}, \t{.sqlite}, \t{.sqlite3}}...
	\end{itemize}
\end{frame}

\begin{frame}{Анонс домашнего задания}
	\begin{itemize}
		\item Вам будет выдан файл с SQL-запросами, которые создают таблицы со странами (структуру разберём) и заполняют их данными.
		\item Вам нужно написать несколько SQL-запросов \t{SELECT}, которые что-то вычисляют.
		\item Тестировать можно на созданных тестовых данных.
		\item Как именно тестировать "--- сейчас покажу.
	\end{itemize}
\end{frame}

\begin{frame}{Консольная утилита}
	\begin{itemize}
		\item Называется sqlite3. Это просто программа, которая умеет выполнять SQL-запросы на БД sqlite.
		\item По умолчанию создаёт пустую БД в памяти.
		\item Можно попросить открыть существующую БД в файле (или создать новый файл).
		\item При помощи перенаправления может выполнять SQL из файла.
		\item SQL-запрос должен заканчиваться точкой с запятой.
	\end{itemize}
\end{frame}

\begin{frame}{Графическая утилита}
	\begin{itemize}
		\item Я выбрал \href{http://sqlitebrowser.org/}{DB Browser for SQLite}.
		\item Иногда проще смотреть на таблице в графической оболочке, чем в консоли.
		\item Может открывать файлы с БД, все изменения идут в памяти.
		\item Можно откатывать изменения кнопкой <<Revert Changes>> до последнего сохранения.
		\item Можно сохранять изменения в файл кнопкой <<Write Changes>>.
		\item Показывает таблицы, их структура, позволяет выполнять произвольные запросы.
	\end{itemize}
\end{frame}

\begin{frame}[fragile]{Python}
\begin{minted}{python}
with sqlite3.Connection("literacy.sqlite3") as db:
  cursor = db.execute("SELECT * FROM Country LIMIT 3")
  print(cursor.description)
  print(list(cursor))
  print(list(cursor))  # Что-нибудь выведет?
\end{minted}
	\begin{itemize}
		\item Терминология очень похожа во всех языках и СУБД.
		\item Обычно в языке есть стандартный интерфейс общения с любыми СУБД.
			А \textit{драйвер} СУБД реализует этот интерфейс в языке.
		\item Сначала мы устанавливаем \textit{соединение} с СУБД.
		\item Результатом запроса является \textit{курсор} "--- это такой итератор по строчкам запроса.
		\item Что возвращают запросы, кроме \t{SELECT} "--- зависит от СУБД.
		\item
			Иногда считается, что не запрос возвращает курсор, а надо
			сначала создать курсор, а потом в нём выполнить запрос.
	\end{itemize}
\end{frame}
