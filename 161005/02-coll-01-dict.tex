\section{Притворяемся коллекцией}
\subsection{Словарь}
\begin{frame}
	\tableofcontents[currentsection,currentsubsection]
\end{frame}

\begin{frame}[fragile]{Словарь}
\begin{minted}{python}
class KeyToPrependedKey:
    def __init__(self, prefix):
        self.prefix = prefix
    def __getitem__(self, name):
        return self.prefix + name
a = KeyToPrependedKey("foo_")
print(a["bar"])  # foo_bar
\end{minted}
	Ещё бывают методы \t{\_\_setitem\_\_}, \t{\_\_delitem\_\_}, \t{\_\_len\_\_}, \t{\_\_contains\_\_}, \t{\_\_reversed\_\_}, \t{\_\_missing\_\_}, \t{\_\_iter\_\_} (см. дальше).
\end{frame}

\begin{frame}[fragile]{Массив со срезами}
	Срезы передаются просто как объект типа \t{slice}:
\begin{minted}{python}
class RangeMeasurer:
    def __getitem__(self, s):
        if isinstance(s, slice):
            return s.stop - s.start
        else:
            return 1
print(RangeMeasurer()[2:4])    # 2
print(RangeMeasurer()[4:2])    # -2
print(RangeMeasurer()[2:4:2])  # 2
print(RangeMeasurer()[:2])     # ???
print(RangeMeasurer()[2:])     # ???
\end{minted}
	\pause
	Все случаи надо либо разбирать руками, либо использовать функцию \t{slice.range()}.
\end{frame}

\begin{frame}[fragile]{Присваивание}
	Оператор \t{=} в Python в общем случае перегрузить нельзя: \t{a = b}
	всегда изменит значение \t{a} на \t{b} копированием ссылки.
	Но можно перегрузить в частных случаях:
\begin{minted}{python}
class KeyPrepender:
    def __init__(self, backend, prefix):
        self.backend = backend
        self.prefix = prefix
    def __getitem__(self, name):
        return self.backend[self.prefix + name]
    def __setitem__(self, name, val):
        self.backend[self.prefix + name] = val
d = {}
cache = KeyPrepender(d, "my_")
cache["foo"] = 10
print(cache["foo"])  # 10
print(d)             # {'my_foo': 10}
\end{minted}

	В C++ наоборот: перегрузить можно только оператор \t{=} в общем случае,
	поэтому возникают прокси-объекты (тут не рассматриваем).
\end{frame}
