\documentclass[utf8,xcolor=table]{beamer}

\usepackage[T2A]{fontenc}
\usepackage[utf8]{inputenc}
\usepackage[english,russian]{babel}
\usepackage{minted}
\usepackage{ulem}
\usepackage{cmap}

\hypersetup{colorlinks,linkcolor=blue,urlcolor=blue}

\mode<presentation>{
	\usetheme{CambridgeUS}
}

\renewcommand{\t}[1]{\ifmmode{\mathtt{#1}}\else{\texttt{#1}}\fi}

\title{Мимикрия, метаклассы, паттерны}
\author{Егор Суворов}
\institute[СПб АУ]{Курс <<Парадигмы и языки программирования>>, подгруппа 3}
\date[05.10.2016]{Среда, 5 октября 2016 года}

\setlength{\arrayrulewidth}{1pt}

\begin{document}

\begin{frame}
\titlepage
\end{frame}

\begin{frame}{План занятия}
	\tableofcontents
\end{frame}

\section{Перегрузка операторов}
\subsection{Мотивация}

\begin{frame}
	\tableofcontents[currentsection]
\end{frame}

\begin{frame}[fragile]{Зачем перегружать?}
	\begin{itemize}
		\item Задание требуемой семантики у своих объектов:	
\begin{minted}{python}
class Foo:
    def __init__(self, value):
        self.value = value
Foo("hello") == Foo("hello")  # False
\end{minted}

		\item Упрощение кода с математическими объектами:
\begin{minted}{python}
# До
res = a.multiply(x).add(b.multiply(y)) \
    .add(c).multiply(5).add(2)
middle = vector1.add(vector2).multiply(0.5)
	
# После
res = (a * x + b * y + c) * 5 + 2
middle = (vector1 + vector2) / 2
\end{minted}
	\end{itemize}
\end{frame}

\begin{frame}[fragile]{Почему не перегружать?}
	\pause
	\begin{itemize}
		\item
			Разное поведение у похожих типов.
			Например: определим \t{/} как целочисленное деление.
			Лучше определить только \t{//}.
			\pause
		\item
\begin{minted}{cpp}
line a = /* ... */, b = /* ... */;
if (a || b) { /* ... */ }
\end{minted}
			\pause
			Можно сказать, что \t{||} возвращает, параллельны ли прямые.
			Полностью изменяется семантика оператора.
			\pause
		\item
			Скрывает сложные операции при чтении кода.
			Может мешать при отладке и поиске медленных мест "--- нет явного вызова функции:
\begin{minted}{python}
a = 10 ** 10000
b = 10 ** 10000
# ...
result = a * b  # Почему же тормозит?
\end{minted}
		\item
			Неочевидное поведение: \t{vec1 * vec2}.
			\pause
			Векторное или скалярное произведение?
	\end{itemize}
\end{frame}

\subsection{Немного магии}
\begin{frame}{Магические методы}
	\begin{itemize}
		\item
			\textit{Магическим} зовётся метод, название которого начинается и заканчивается на \t{\_\_}.
			Например, \t{\_\_init\_\_} или \t{\_\_str\_\_}.
		\item Ничего магического, кроме предназначения, в них нет.
		\item Напрямую их вызывать не стоит!
		\item Перечислены в документации по группам. Объекты могут прикидываться:
			\begin{enumerate}
				\item \href{https://docs.python.org/3/reference/datamodel.html\#emulating-numeric-types}{Числами}.
				\item \href{https://docs.python.org/3/reference/datamodel.html\#object.\_\_lt\_\_}{Чем-то, что можно сравнивать}.
				\item \href{https://docs.python.org/3/reference/datamodel.html\#emulating-callable-objects}{Функциями}.
				\item \href{https://docs.python.org/3/reference/datamodel.html\#emulating-container-types}{Коллекциями} (массив, словарь, множество...).
				\item \href{https://docs.python.org/3/library/stdtypes.html\#typeiter}{Итераторами} (обслуживают цикл \t{for}).
				\item \href{https://docs.python.org/3/reference/datamodel.html\#with-statement-context-managers}{Чем-что, что можно автоматически закрывать} (файл, сетевое соединение).
				\item И ещё много чем.
			\end{enumerate}
		\item Какая-то информация гуглится и \href{https://pythonworld.ru/osnovy/peregruzka-operatorov.html}{на русском}.
	\end{itemize}
\end{frame}

\subsection{Синтаксис арифметики}

\begin{frame}
	\tableofcontents[currentsection,currentsubsection]
\end{frame}

\begin{frame}[fragile]{Как перегружать}
\begin{minted}{python}
class Natural:
    def __init__(self, value):
        assert value >= 1
        self.value = value
    def __add__(self, other):
        return Natural(self.value + other.value)
    def __sub__(self, other):
        return Natural(self.value - other.value)
    def  __repr__(self):  # Почти __str__
        return "Natural({})".format(self.value)
print(Natural(4) + Natural(3))  # Natural(7)
print(Natural(4) - Natural(3))  # Natural(1)
print(Natural(4).__sub__(Natural(3)))  # Не надо так!
print(Natural(4) - Natural(4))  # AssertionError
\end{minted}
\end{frame}

\begin{frame}[fragile]{Разные типы-1}
\begin{minted}{python}
print(Natural(4) + 3)           # AttributeError
def better_add(self, other):
    if isinstance(other, Natural):
        return Natural(self.value + other.value)
    elif isinstance(other, int):
        return Natural(self.value + other)
    else:
        return NotImplemented
Natural.__add__ = better_add

print(Natural(4) + Natural(3))  # Natural(7)
print(Natural(4) + 3)           # Natural(7)
print(Natural(4) + "3")         # TypeError
print(3 + Natural(4))           # TypeError?
\end{minted}
\end{frame}

\begin{frame}[fragile]{Разные типы-2}
\begin{minted}{python}
print(3 + Natural(4))  # У int нет метода __add__ для Natural
int.__add__ = None     # И задать часто нельзя. И не надо.
Natural.__radd__ = Natural.__add__

print(Natural(4) + 3)           # Natural(7)
print(3 + Natural(4))           # Natural(7)
\end{minted}

	При вычислении выражения \t{a + b}:
	\begin{enumerate}
		\item Вызывается \t{a.\_\_add\_\_(b)}.
		\item Если метод найден и не вернули \t{NotImplemented} "--- успех.
		\item Иначе, если \t{a} и \t{b} разных типов, вызывается \t{b.\_\_radd\_\_(a)}\footnote{\textbf{r}everse \textbf{a}dd}.
		\item Если не помогло "--- неуспех.
	\end{enumerate}

	Также есть методы \t{\_\_rsub\_\_}, \t{\_\_rmul\_\_} и другие.
\end{frame}

\begin{frame}[fragile]{Разные типы-3}
\begin{minted}{python}
class Foo:
    def __add__(self, other):
        print("add")
        return NotImplemented
    def __radd__(self, other):
        print("radd")
        return self

Natural(3) + Foo()  # radd
Foo() + Natural(3)  # add, TypeError
\end{minted}
\end{frame}

\subsection{Синтаксис сравнений}
\begin{frame}
	\tableofcontents[currentsection,currentsubsection]
\end{frame}

\begin{frame}[fragile]{Сравнения-1}
\begin{minted}{python}
class Natural:
    def __init__(self, value):
        self.value = value
    def __lt__(self, other):  # Less than
        return self.value < other.value
    def __le__(self, other):  # Less or equal
        return self.value <= other.value
    def __eq__(self, other):
        return self.value == other.value
ONE, TWO = map(Natural, [1,2])
print(ONE < TWO, ONE > TWO)    # True False
print(ONE <= TWO, ONE >= TWO)  # True False
print(ONE == TWO, ONE != TWO)  # False True
print(ONE == ONE, ONE != ONE)  # True False
\end{minted}
\end{frame}

\begin{frame}[fragile]{Сравнения-2}
	\begin{itemize}
		\item
			Операторы \t{\_\_lt\_\_} и \t{\_\_gt\_\_} считаются отражениями друг друга.
			Почти как \t{\_\_add\_\_} и \t{\_\_radd\_\_}.
		\item
			Оператор \t{\_\_ne\_\_} по умолчанию берёт отрицание от \t{\_\_eq\_\_}.
		\item
			Оператор \t{\_\_lt\_\_} автоматически из \t{\_\_le\_\_} и \t{\_\_eq\_\_} не выводится.
	\end{itemize}
	Можно использовать \textit{декоратор} \href{https://docs.python.org/3/library/functools.html\#functools.total\_ordering}{\t{total\_ordering}}:
\begin{minted}{python}
from functools import total_ordering
@total_ordering  # Магия.
class Natural:
    # Надо задать метод __eq__ и один из четырёх сравнивающих.
    #
    # Остальное сгенерируется.
\end{minted}
\end{frame}

\begin{frame}[fragile]{Хэш-таблицы-1}
\begin{minted}{python}
a = { Natural(1): 1 }  # TypeError: unhashable type

class Natural:
    def __init__(self, value):
        self.value = value
    def __eq__(self, other):
        return self.value == other.value
    def __hash__(self):
        return hash(self.value)
    def __repr__(self):
        return "Natural({})".format(self.value)

a = {Natural(x): x for x in range(5)}
print(a)
\end{minted}
\end{frame}

\begin{frame}{Хэш-таблицы-2}
	\begin{itemize}
		\item
			\t{\_\_hash\_\_} вызывается функцией \t{hash}, когда элемент кладут в хэш-таблицу.
			Должна вернуть \t{int}.
		\item
			Требование: если \t{a == b}, то \t{hash(a) == hash(b)} (в обратную сторону необязательно).
		\item
			Если определён \t{\_\_hash\_\_}, то обязательно определить \t{\_\_eq\_\_}.
		\item
			Для помещения в хэш-таблицу методы \t{\_\_lt\_\_} необязательны.
		\item
			В языке Java идеология похожа: методы \t{equals()} и \t{hashCode()}.
		\item
			Если объект может измениться (\textit{мутабельный}), то \t{\_\_hash\_\_} определять не стоит.
			Почему?
			\pause
			Потому что если он изменится, пока лежит в хэш-таблице, она об этом не узнает и сломается.
	\end{itemize}
\end{frame}

\begin{frame}[fragile]{Хэш-таблицы-3}
	Чем плоха реализация строки ниже?
\begin{minted}{python}
class Str:
    def __init__(self, value): self.value = value
    def __eq__(self, other): return self.value == other.value
    def __hash__(self):
        result = 0
        for c in self.value:
            result = result * 239017 + ord(c)
        return result
\end{minted}
	\pause
\begin{minted}{python}
print(hash("a" * 1000))
print(hash(Str("a" * 1000)))
print(hash("a" * 100000))
print(hash(Str("a" * 100000)))
\end{minted}
	\pause
	Тормозит, потому что в Python \t{int} автоматически преобразуется в длинную арифметику,
	а не переполняется.
\end{frame}

\begin{frame}[fragile]{Упражнение}
	Реализуйте класс \t{Natural} для хранения натуральных чисел с поддержкой сложения, умножения (в том числе с int), хэширования, вывода на экран и сравнений:
\begin{minted}{python}
class Natural:
    # ... ваш код здесь ...

# Тесты:
l = list(map(Natural, [5, 2, 3, 4, 1, 4]))
print(Natural(2) + Natural(3))
print(20 * Natural(100) + 5)
print(sum(l))
print(sorted(l))
d = {Natural(1): 10, Natural(2): 20}
print(list(d.keys()))  # [Natural(1), Natural(2)]
\end{minted}
\end{frame}

\subsection{Наследование}
\begin{frame}
	\tableofcontents[currentsection,currentsubsection]
\end{frame}

\begin{frame}[fragile]{Беда?}
\begin{minted}{python}
class Point:
    def __init__(self, x, y): self.x, self.y = x, y
    def __eq__(self, other):
        return (self.x, self.y) == (other.x, other.y)
class PointWithId(Point):
    def __init__(self, x, y, id):
        super(PointWithId, self).__init__(x, y)
        self.id = id
    def __eq__(self, other):  # Принцип Лисков?
        return ((self.x, self.y, self.id) ==
                (other.x, other.y, other.id))
base, child = Point(1, 2), PointWithId(1, 2, 3)
print(base.__eq__(child))            # True
print(child.__eq__(base))            # AttributeError
print(base == child, child == base)  # ???
\end{minted}
\end{frame}

\begin{frame}[t]{Беда!}
	Неясно, как сравнивать объекты разных типов на равенство в общем случае:
	\begin{enumerate}
		\item Можно запретить сравнение объектов разных типов, но нарушится
			\only<-1>{???}\only<2->{принцип подстановки.}
		\item Можно сравнивать только по общим полям, но тогда нарушится
			\only<-2>{???}
			\only<3->{
			транзитивность:
			\[(child_1 = base) \land (base = child_2) \not\Rightarrow child_1=child_2\]
			}
	\end{enumerate}
	\only<4->{
	В любом случае, суперкласс ничего про детей не знает, поэтому вызывать его метод скорее бессмысленно.
	Python всегда вызовет метод подкласса (будь то \t{\_\_eq\_\_}, \t{\_\_le\_\_} или \t{\_\_ge\_\_}).

	В других языках может быть по-другому!

	Надёжнее всего считать объекты разных типов разными.
	}
\end{frame}

\begin{frame}{Вообще беда}
	С сортировкой ещё хуже.
	Надо определить какой-то линейный порядок на всех объектах:
	не только транзитивность, но ещё и согласованность с равенством, например:
	\[ a \neq b \iff (a < b) \lor (a > b) \]

	Как разрулить в общем случае?
	\pause
	Запретить сравнивать объекты разных типов!

	Как тогда не нарушить принцип подстановки?
	\pause
	Запретить сравнивать объекты типа <<суперкласс>>, если только не известно заведомо, что они одного типа.
\end{frame}

\subsection{Притворяемся функцией}
\begin{frame}
	\tableofcontents[currentsection,currentsubsection]
\end{frame}

\begin{frame}[fragile]{Притворяемся функцией}
\begin{minted}{python}
class Summer:  # Сумма, а не лето :(
    def __init__(self, k):
        self.k = k
    def __call__(self, *args):
        return self.k * sum(args)
s = Summer(3)
print(s())       # 0
print(s(1))      # 3
print(s(1, 10))  # 33
\end{minted}
	Всё, что имеет метод \t{\_\_call\_\_}, может быть вызвано.
	И наоборот:
\begin{minted}{python}
def foo(): print("foo")
print(foo.__call__)
foo.__call__()
print(foo.__call__.__call__)
\end{minted}
\end{frame}

\begin{frame}{Но зачем?}
	\begin{itemize}
		\item
			Объект каком-то смысле представляет собой функцию.
			Например: <<преобразование плоскости>>, <<логгер>> или <<выражение от одной переменной>>.
		\item
			В некоторых других языках это единственный способ сделать функцию с некоторым внутренним состоянием (кроме глобальных переменных).
		\item
			Все проблемы с перегрузкой операторов остаются.
			Не злоупотребляйте!
	\end{itemize}
\end{frame}

\section{Притворяемся коллекцией}
\subsection{Словарь}
\begin{frame}
	\tableofcontents[currentsection,currentsubsection]
\end{frame}

\begin{frame}[fragile]{Словарь}
\begin{minted}{python}
class KeyToPrependedKey:
    def __init__(self, prefix):
        self.prefix = prefix
    def __getitem__(self, name):
        return self.prefix + name
a = KeyToPrependedKey("foo_")
print(a["bar"])  # foo_bar
\end{minted}
	Ещё бывают методы \t{\_\_setitem\_\_}, \t{\_\_delitem\_\_}, \t{\_\_len\_\_}, \t{\_\_contains\_\_}, \t{\_\_reversed\_\_}, \t{\_\_missing\_\_}, \t{\_\_iter\_\_} (см. дальше).
\end{frame}

\begin{frame}[fragile]{Массив со срезами}
	Срезы передаются просто как объект типа \t{slice}:
\begin{minted}{python}
class RangeMeasurer:
    def __getitem__(self, s):
        if isinstance(s, slice):
            return s.stop - s.start
        else:
            return 1
print(RangeMeasurer()[2:4])    # 2
print(RangeMeasurer()[4:2])    # -2
print(RangeMeasurer()[2:4:2])  # 2
print(RangeMeasurer()[:2])     # ???
print(RangeMeasurer()[2:])     # ???
\end{minted}
	\pause
	Все случаи надо либо разбирать руками, либо использовать функцию \t{slice.range()}.
\end{frame}

\begin{frame}{Присваивание-1}
	Оператор \t{=} в Python в общем случае перегрузить нельзя: \t{a = b}
	всегда изменит значение \t{a} на \t{b} копированием ссылки.

	Но можно перегрузить в частных случаях, например, когда мы пишем \t{a.foo = b}
	или \t{a["foo"] = b}.

	В C++ наоборот: перегрузить можно только оператор \t{=} в общем случае,
	поэтому возникают прокси-объекты (тут не рассматриваем).
\end{frame}

\begin{frame}[fragile]{Присваивание-2}
\begin{minted}{python}
class KeyPrepender:
    def __init__(self, backend, prefix):
        self.backend = backend
        self.prefix = prefix
    def __getitem__(self, name):
        return self.backend[self.prefix + name]
    def __setitem__(self, name, val):
        self.backend[self.prefix + name] = val
d = {}
cache = KeyPrepender(d, "my_")
cache["foo"] = 10
print(cache["foo"])  # 10
print(d)             # {'my_foo': 10}
\end{minted}

\end{frame}

\subsection{Итераторы}
\begin{frame}
	\tableofcontents[currentsection,currentsubsection]
\end{frame}

\begin{frame}[fragile]{Пример итератора}
\begin{minted}{python}
class CountDownIterator:
    def __init__(self, start): self.value = start
    def __iter__(self):
        return self  # Так надо.
    def __next__(self):
        if self.value < 1:
            raise StopIteration
        self.value -= 1
        return self.value + 1
v = CountDownIterator(5)
print(v)
print(next(v))  # 5
print(next(v))  # 4
\end{minted}
\end{frame}

\begin{frame}[fragile]{Коллекции и итераторы}
	Всё, у чего есть метод \t{\_\_iter\_\_}, можно запихнуть в цикл \t{for} и другие интересные места:
\begin{minted}{python}
a = [1, 2, 3]
print([ x for x in a ])  # [1, 2, 3]
it = iter(a)  # Вызывает __iter__.
print(next(it))  # 1
print(next(it))  # 2
print(next(it))  # 3
print(next(it))  # StopIteration

a = CountDownIterator(5)
print(list(a))  # 5 4 3 2 1
print(list(a))  # ???
\end{minted}
\end{frame}

\begin{frame}{Что произошло}
	\begin{itemize}
		\item Если коллекция закончилась, итератор должен вызвать \t{raise~StopIteration}.
		\item Итератор "--- штука одноразовая, переиспользовать нельзя.
		\item Метод \t{\_\_iter\_\_} у коллекций возвращает \textbf{новый} итератор, указывающий на начало коллекции.
		\item Метод \t{\_\_iter\_\_} есть у каждого итератора, чтобы их можно было использовать там же, где и коллекции.
		\item В других языках интерфейсы коллекции и итератора разнесены более явно.
	\end{itemize}
\end{frame}

\begin{frame}{StopIteration}
	\t{StopIteration} "--- это так называемое \textit{исключение}.

	\begin{itemize}
		\item Исключения "--- один из механизмов обработки ошибок (исключительных ситуаций):
			\begin{enumerate}
				\item Произошла ошибка.
				\item Создали объект класса \textit{исключение} (или подкласса: \t{TypeError}, \t{KeyError}, ...).
				\item \t{Кинули} его (сленг) командой \t{raise SomeException()}.
				\item Исключение пошло вверх по стеку вызовов до ближайшего обработчика, соответствующего типа.
				\item Обработчик решает, что делать с исключением.
			\end{enumerate}
		\item Можно воспринимать как такой <<\t{return} из всех функций сразу до ближайшего обработчика>>.
		\item Более подробно пока не рассматриваем.
	\end{itemize}
\end{frame}

\begin{frame}[fragile]{Как ловить StopIteration}
\begin{minted}{python}
def print_next(it):
    print(next(it))
def print_two_next(it):
    print_next(it)
    print_next(it)
    print("Printed two items")
try:  # Начало блока, где может вылететь исключение.
    print_two_next(iter([1]))
    print("Finished")
except StopIteration:  # Обработчик исключения StopIteration.
    print("Stopped")
\end{minted}
\end{frame}

\begin{frame}[fragile]{Упражнение}
	Напишите <<свою>> реализацию цикла \t{for}:
\begin{minted}{python}
def foreach(items, callback):
    # Перепишите эту функцию без цикла for.
    for item in items:
        callback(item)

# Примеры.
foreach([1, 2, 3], print) # 1 2 3
foreach([1, 2, 3], lambda x: print(x, end=";")) # 1;2;3;
print()
foreach(CountDownIterator(3), print) # 3 2 1

foreach({"a": False, "b": True}.items(), print)
# ('a', False) ('b', True) в любом порядке.
\end{minted}
\end{frame}

\section{Менеджеры контекста}

\begin{frame}
	\tableofcontents[currentsection,currentsubsection]
\end{frame}

\begin{frame}[fragile]{Кто такие}
\begin{minted}{python}
with open("file.txt", "r") as f:
    f.read()
# Почти то же самое, что:
f = open("file.txt", "r")
f.__enter__()
f.read()
f.__exit__()
\end{minted}
	Тут \t{f} "--- \href{https://docs.python.org/3/library/stdtypes.html\#typecontextmanager}{\textit{менеджер контекста}}, потому что он реализует методы \t{\_\_enter\_\_} и \t{\_\_exit\_\_}.
	Жизненный цикл:
	\begin{enumerate}
		\item Создали менеджер, вызвался \t{\_\_init\_\_}.
		\item Вошли в блок \t{with}, вызвался \t{\_\_enter\_\_}.
		\item Вышли из блока (в том числе при помощи \t{return}), вызвался \t{\_\_exit\_\_}.
	\end{enumerate}
\end{frame}

\begin{frame}{Зачем}
	Примеры:
	\begin{itemize}
		\item Файлы: открываем и закрываем.
		\item Блокировки: их можно создавать, а дальше захватывать для эксклюзивного доступа (и потом отпускать).
		\item Папка с временными файлами: создать, потом почистить.
		\item Смена текущей папки на время работы функции.
	\end{itemize}
	Применение:
	\begin{itemize}
		\item Когда ваш объект использует какие-то ресурсы, которые надо закрывать: файлы, сетевые соединения.
		\item Менеджер может предполагать, что он создаётся и сразу используется в блоке \t{with}.
			Тогда разница между \t{\_\_init\_\_} и \t{\_\_enter\_\_} тонкая.
			Пример: файл.
		\item
			Менеджер может предполагать, что за время жизни его могут использовать в разных блоках \t{with}.
			Тогда надо различать \t{\_\_init\_\_} и \t{\_\_enter\_\_}.
			Пример: блокировка.
	\end{itemize}
\end{frame}

\begin{frame}[fragile]{Упражнение}
	Напишите менеджер контекста для смены текущей папки (функция \t{os.chdir()}):
\begin{minted}{python}
class ChangeDir:
    # ... ваш код здесь ...

print(os.getcwd())      # /home/foo/bar
with ChangeDir(".."):   # as можно опускать
    print(os.getcwd())  # /home/foo
print(os.getcwd())      # /home/foo/bar
\end{minted}
\end{frame}

\section{Метаклассы}

\begin{frame}
	\tableofcontents[currentsection,currentsubsection]
\end{frame}

\begin{frame}[t,fragile]{Идея}
\begin{minted}{python}
class A:
    pass
a = A()
print(type(a))  # <class '__main__.A'>
print(type(A))  # type
\end{minted}
	\begin{itemize}
		\item В Python всё динамическое, и почти всё в некотором смысле "--- объект (имеет методы и свойства, которые можно менять).
		\item Так зачем разделять <<классы>> и <<объекты>> в языке, если они ничем по поведению не отличаются, только по использованию?
		\only<2->{
		\item Классы "--- это тоже объекты класса \t{type}.
		\item \t{type} "--- это тоже класс. И тоже объект класса \only<2>{...}\only<3->{\t{type}. Этакая рекурсия.
		\item Обычно метаклассы в программах писать и использовать не требуется.
		}}
	\end{itemize}
\end{frame}

\begin{frame}{Замечание}
	В других языках я ничего похожего не видел, кроме, пожалуй JavaScript (там свои тараканы).

	Обычно есть чёткое разделение на <<класс>> (статическое описание структуры объектов) и <<объект>>.
\end{frame}

\begin{frame}[fragile]{Зачем}
	Используется, когда хочется изменить стандартное поведение слова \t{class}:
	\begin{itemize}
		\item Автоматически сгенерировать новые методы.
		\item Проверить, что в классе определены все нужные методы.
		\item Зарегистрировать класс где-нибудь при создании (например, если мы хотим знать все классы, которые у нас есть).
		\item Автоматически обернуть все методы в декоратор (для логов, для замеров времени).
	\end{itemize}
\end{frame}

\begin{frame}[fragile]{Применение}
	Обычно метаклассы незаметны; авторы библиотеки просто говорят, как надо писать код:
\begin{minted}{python}
from django.db import models
class Person(models.Model):
    first_name = models.CharField(max_length=30)
    last_name = models.CharField(max_length=30)
\end{minted}
	За этими тремя строками скрывается:
	\begin{itemize}
		\item Автоматическая генерация таблицы в базе данных.
		\item Создание страницы сайта для редактирования.
		\item Создание конструктора с параметрами \t{first\_name} и \t{last\_name}.
		\item Создание методов для загрузки и сохранения объектов в БД.
		\item Создание методов для проверки корректности данных.
		\item Создание методов \t{\_\_eq\_\_}, \t{\_\_str\_\_} и других.
	\end{itemize}
\end{frame}

\begin{frame}[fragile]{Пример}
\begin{minted}{python}
class NamedTupleMeta(type):
    # Тут используют метод __new__ вместо __init__.
    def __new__(cls, name, bases, dct):  # cls, не self.
        print(cls, name, bases, dct)
        attrs = dct["attrs"]
        def init(self, **kwargs):
            for key in kwargs:
                assert key in attrs, "Unknown key: " + key
                setattr(self, key, kwargs[key])
        dct["__init__"] = init
        del dct["attrs"]
        return super().__new__(cls, name, bases, dct)

class Person(metaclass=NamedTupleMeta):
    attrs = ["first_name", "last_name"]
\end{minted}
\end{frame}

\section{Паттерны}
\subsection{Классификация}

\begin{frame}
	\tableofcontents[currentsection,currentsubsection]
\end{frame}

\begin{frame}{Что такое паттерн}
	\begin{itemize}
		\item
			\textit{Паттерн} (или \textit{шаблон проектирования}) "--- это какая-то стандартная конструкция для решения каких-то архитектурных задач.
		\item
			Другими словами: какие абстракции и интерфейсы полезно использовать в каких ситуациях.
		\item
			Некоторые паттерны идут <<от капитана>> (можно назвать <<очевидным здравым смыслом>>), некоторые более хитры.
		\item
			Часто может казаться, что без паттерна легко обойтись, потому что в программе нужен очень частный случай.
			Зато если требования поменяются "--- можно огрести.
		\item
			Легко перегнуть палку: <<абстрактные фабрики абстрактных фабрик>> и прочие радости.
		\item
			Мы пройдёмся очень поверхностно.
			Вообще есть большие книжки (от <<банды четырёх>>) и курсы, где про это рассказывают.
		\item
			Часто называются по-английски.
	\end{itemize}
\end{frame}

\begin{frame}{Поведенческие шаблоны}
	Рассказывают, как объекты могут между собой взаимодействовать.

	Примеры:
	\begin{itemize}
		\item
			\textit{Итератор} (\textit{Iterator}): уже познакомились.
			Позволяет итерироваться по произвольным коллекциям.
		\item
			\textit{Наблюдатель} (\textit{Observer}): если некоторые объекты могут рассылать события, а некоторые
			должны на них реагировать, то можно создать интерфейс <<наблюдатель за событием>> и вспомогательные классы
			для рассылки событий.
			Тогда наблюдателю надо лишь добавиться в нужный список, а инициатору события "--- вызвать метод <<оповести наблюдателей из списка>>.
		\item
			\textit{Посетитель} (\textit{Visitor}): будет в домашнем задании, разберём позже.
	\end{itemize}
\end{frame}

\begin{frame}{Структурные шаблоны}
	Рассказывают, как компоновать между собой классы и объекты.

	Примеры:
	\begin{itemize}
		\item
			\textit{Адаптер} (\textit{Adapter}): если у нас есть класс с интерфейсом $A$, а нам нужно передать куда-то класс с другим интерфейсом $B$
			(другие название), то можно создать класс, который просто будет конвертировать вызовы интерфейса $B$ в вызовы $A$.
		\item
			\textit{Компоновщик} (\textit{Composite}): если нам часто нужно совершать одинаковые операции над разными объектами (например, отрисовать элементы
			окна на экране), то их можно объединить в коллекцию, которая предоставляет общий интерфейс для этих объектов.
	\end{itemize}
\end{frame}

\begin{frame}[fragile]{Порождающие шаблоны}
	Рассказывают, как создавать и компоновать объекты в коде.
	Примеры:
	\begin{itemize}
		\item
			\textit{Строитель} (\textit{Builder}): если есть объект с очень сложным конструктором, то можно создать
			промежуточный объект, который будет <<накапливать>> в себе параметры конструктора, а потом создаст объект:
\begin{minted}{python}
def create_button():
    builder = ButtonBuilder()
    builder.set_text("Кнопка")
    if some_complex_condition():
        builder.set_image("image.png")
    return builder.build()
\end{minted}
		\item
			\textit{Абстрактная фабрика}  (\textit{Abstract factory}): если объекты постоянно требуют какого-то включения в систему, то можно
			не вызывать конструкторы напрямую, а выделить кусок системы, который будет
			правильным образом конструировать объекты.
	\end{itemize}
\end{frame}


\end{document}
