\section{Менеджеры контекста}

\begin{frame}
	\tableofcontents[currentsection,currentsubsection]
\end{frame}

\begin{frame}[fragile]{Кто такие}
\begin{minted}{python}
with open("file.txt", "r") as f:
    f.read()
# Почти то же самое, что:
f = open("file.txt", "r")
f.__enter__()
f.read()
f.__exit__()
\end{minted}
	Тут \t{f} "--- \href{https://docs.python.org/3/library/stdtypes.html\#typecontextmanager}{\textit{менеджер контекста}}, потому что он реализует методы \t{\_\_enter\_\_} и \t{\_\_exit\_\_}.
	Жизненный цикл:
	\begin{enumerate}
		\item Создали менеджер, вызвался \t{\_\_init\_\_}.
		\item Вошли в блок \t{with}, вызвался \t{\_\_enter\_\_}.
		\item Вышли из блока (в том числе при помощи \t{return}), вызвался \t{\_\_exit\_\_}.
	\end{enumerate}
\end{frame}

\begin{frame}{Зачем}
	Примеры:
	\begin{itemize}
		\item Файлы: открываем и закрываем.
		\item Блокировки: их можно создавать, а дальше захватывать для эксклюзивного доступа (и потом отпускать).
		\item Папка с временными файлами: создать, потом почистить.
		\item Смена текущей папки на время работы функции.
	\end{itemize}
	Применение:
	\begin{itemize}
		\item Когда ваш объект использует какие-то ресурсы, которые надо закрывать: файлы, сетевые соединения.
		\item Менеджер может предполагать, что он создаётся и сразу используется в блоке \t{with}.
			Тогда разница между \t{\_\_init\_\_} и \t{\_\_enter\_\_} тонкая.
			Пример: файл.
		\item
			Менеджер может предполагать, что за время жизни его могут использовать в разных блоках \t{with}.
			Тогда надо различать \t{\_\_init\_\_} и \t{\_\_enter\_\_}.
			Пример: блокировка.
	\end{itemize}
\end{frame}

\begin{frame}[fragile]{Упражнение}
	Напишите менеджер контекста для смены текущей папки (функция \t{os.chdir()}):
\begin{minted}{python}
class ChangeDir:
    # ... ваш код здесь ...

print(os.getcwd())      # /home/foo/bar
with ChangeDir(".."):   # as можно опускать
    print(os.getcwd())  # /home/foo
print(os.getcwd())      # /home/foo/bar
\end{minted}
\end{frame}
