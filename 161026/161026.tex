\documentclass[utf8,xcolor=table]{beamer}

\usepackage[T2A]{fontenc}
\usepackage[utf8]{inputenc}
\usepackage[english,russian]{babel}
\usepackage{minted}
\usepackage{ulem}
\usepackage{cmap}
\usepackage{multirow}

\hypersetup{colorlinks,linkcolor=blue,urlcolor=blue}

\mode<presentation>{
	\usetheme{CambridgeUS}
}

\renewcommand{\t}[1]{\ifmmode{\mathtt{#1}}\else{\texttt{#1}}\fi}

\title{Функциональное программирование}
\author{Егор Суворов}
\institute[СПб АУ]{Курс <<Парадигмы и языки программирования>>, подгруппа 3}
\date[19.10.2016]{Среда, 19 октября 2016 года}

\setlength{\arrayrulewidth}{1pt}

\begin{document}

\begin{frame}
\titlepage
\end{frame}

\begin{frame}{План занятия}
	\tableofcontents
\end{frame}

\input{01-intro-01-imperative}
\input{01-intro-02-decl-func}
%\input{01-intro-03-no-loops}
%\begin{frame}[t,fragile]{Упражнения на рекурсию}
	Факториал:
\begin{minted}{haskell}
fac 0 = 1
fac n = n * fac (n - 1)
\end{minted}
	Степень двойки: \pause
\begin{minted}{haskell}
powerOfTwo 0 = 1
powerOfTwo n = 2 * powerOfTwo (n - 1)
\end{minted}
	Числа Фибоначчи: \pause
\begin{minted}{haskell}
fib n = fib' 0 1 n
-- Одна итерация цикла for
fib' a b 0 = a
fib' a b n = fib' b (a + b) (n - 1)
\end{minted}
	Мы увидели:
	\begin{itemize}
		\item Pattern matching (вместо \t{if} пишем <<шаблон>> для аргумента функции).
		\item Апостроф является корректным символом в названии функции.
	\end{itemize}
\end{frame}

\begin{frame}[t,fragile]{Жизнь без циклов-2}
	Как теперь функцию \t{sum}? \pause
\begin{minted}{haskell}
sum' (x:xs) = x + sum xs
sum' _ = 0
\end{minted}
	\begin{itemize}
		\item Иногда можно делать сложный pattern matching вроде \t{x:xs} (список, первый элемент которого "--- \t{x}, а хвост "--- \t{xs}).
		\item Шаблоны проверяются сверху вниз.
		\item Вместо имени переменной можно написать \t{\_}.
	\end{itemize}
	А как написать в императивном стиле? \pause
\begin{minted}{haskell}
sum' xs = sum'' 0 xs
sum'' a [] = a
sum'' a (x:xs) = sum'' (a + x) xs
\end{minted}
\end{frame}

%\input{02-haskell-02-high-order}
%\input{02-haskell-03-lazy}
%\input{02-haskell-04-rakes}
%\input{03-summary-01-pros-cons}
%\input{03-summary-02-other-langs}
%\input{04-links}

\end{document}
