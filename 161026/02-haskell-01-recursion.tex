\begin{frame}[t,fragile]{Упражнения на рекурсию}
	Факториал:
\begin{minted}{haskell}
fac 0 = 1
fac n = n * fac (n - 1)
\end{minted}
	Степень двойки: \pause
\begin{minted}{haskell}
powerOfTwo 0 = 1
powerOfTwo n = 2 * powerOfTwo (n - 1)
\end{minted}
	Числа Фибоначчи: \pause
\begin{minted}{haskell}
fib n = fib' 0 1 n
-- Одна итерация цикла for
fib' a b 0 = a
fib' a b n = fib' b (a + b) (n - 1)
\end{minted}
	Мы увидели:
	\begin{itemize}
		\item Pattern matching (вместо \t{if} пишем <<шаблон>> для аргумента функции).
		\item Апостроф является корректным символом в названии функции.
	\end{itemize}
\end{frame}

\begin{frame}[t,fragile]{Жизнь без циклов-2}
	Как теперь функцию \t{sum}? \pause
\begin{minted}{haskell}
sum' (x:xs) = x + sum xs
sum' _ = 0
\end{minted}
	\begin{itemize}
		\item Иногда можно делать сложный pattern matching вроде \t{x:xs} (список, первый элемент которого "--- \t{x}, а хвост "--- \t{xs}).
		\item Шаблоны проверяются сверху вниз.
		\item Вместо имени переменной можно написать \t{\_}.
	\end{itemize}
	А как написать в императивном стиле? \pause
\begin{minted}{haskell}
sum' xs = sum'' 0 xs
sum'' a [] = a
sum'' a (x:xs) = sum'' (a + x) xs
\end{minted}
\end{frame}
