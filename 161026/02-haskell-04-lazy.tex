\subsection{Ленивые вычисления}

\begin{frame}
	\tableofcontents[currentsection,currentsubsection]
\end{frame}

\begin{frame}[fragile]{Идея}
	\begin{itemize}
		\item В классических языках аргументы сначала вычисляются, потом передаются функции: в вызове \t{foo(bar())} сначала вычислится \t{bar()}, а результат передадут в \t{foo()}.
		\item В Haskell наоборот: если функции неважно, что именно ей передали в качестве аргумента "--- это нечто не будет вычислено.
		\item Другими словами, вычисления производят только от плохой жизни: если по-другому никак.
		\item Например, если надо аргумент сравнить с шаблоном.
		\item В функции \t{fst} второй аргумент вычисляться никогда не будет:
\begin{minted}{haskell}
fst x y = x
\end{minted}
		\item А в функции \t{if'} будет вычислен только нужный аргумент:
\begin{minted}{haskell}
if' True  x _ = x
if' False _ y = y
\end{minted}
		\item Приходится тащить вместо невычисленного выражения инструкцию, как его получить.
	\end{itemize}
\end{frame}

\begin{frame}[fragile]{Бесконечные структуры}
	\begin{itemize}
		\item Напоминание: \t{[1, 2, 3]} "--- это сахар для \t{1:2:3:[]}.
		\item Попробуем завести бесконечный список из едини:
\begin{minted}{haskell}
ones = 1:1:1:1:1:1:...
\end{minted}
		\item Так как бесконечную строчку написать не можем, придётся заметить рекурсию:
\begin{minted}{haskell}
ones = 1:ones
\end{minted}
		\item Этого уже хватит для вычисления, попробуйте \t{take 10 ones}.
		\item Как работает: хвост списка вычисляется только тогда, когда он реально нужен.
		\item Поэтому если никто не залез дальше второго элемента, то третий и последующие не будут вычислены, а будет просто храниться, как их можно в случае чего получить.
	\end{itemize}
\end{frame}

\begin{frame}[fragile]{Подробный пример}
\begin{minted}{haskell}
take 0 _ = []
take n (x:xs) = x:(take (n - 1) xs)
ones = 1:ones
take 3 ones = take 3 (1:ones) = 1:(take (3-1) ones)
take 2 ones = take 2 (1:ones) = 1:(take (2-1) ones)
take 1 ones = take 1 (1:ones) = 1:(take (1-1) ones)
take 0 ones = []
take 3 ones = 1:1:1:[] = [1, 1, 1]
\end{minted}
\end{frame}

\begin{frame}[fragile]{Пример похитрее}
\begin{minted}{haskell}
iota n = n:(iota (n + 1))
take 3 (iota 5) = take 3 (5:iota 6) =
5:(take 2 (iota 6)) = 5:(take 2 (6:iota 7)) =
5:6:(take 1 (iota 7)) = 5:6:(take 1 (7:iota 8)) = 
5:6:7:(take 0 (iota 8)) = 5:6:7:[]
iota n = [n, n + 1, n + 2, ...]
\end{minted}
	Практически так реализован синтаксический сахар \t{[5..]}.
\end{frame}

\begin{frame}[fragile]{Числа Фибоначчи}
	Составляем уравнение на список из чисел Фибоначчи:
\begin{minted}{haskell}
fib   = [1, 1, 2, 3, 5, 8, ...]
0:fib = [0, 1, 1, 2, 3, 5, 8, ...]
zipWith (+) fib (0:fib) = 
        [1, 2, 3, 5, 8, 13, ...]
1:zipWith (+) fib (0:fib) =
        [1, 1, 2, 3, 5, 8, ...]
fib = 1:zipWith (+) fib (0:fib)  -- Определение в Haskell
\end{minted}
\end{frame}

\begin{frame}[fragile]{Вычисление Фибоначчи}
\begin{minted}{haskell}
fib = 1:zipWith (+) fib (0:fib)  -- Определение в Haskell
\end{minted}
	\begin{itemize}
		\item Изначально знаем только \t{fib !! 0}.
		\item Когда надо вычислить \t{fib !! 1}, надо вычислить нулевой элемент от \t{zipWith}.
		\item А для этого надо знать нулевые элементы от \t{fib} и \t{0:fib}, оба знаем.
		\item Значит, их сумма и есть \t{fib !! 1}.
		\item Когда потребуется вычислить \t{fib !! 2}, надо будет вычислить первый элемент от \t{zipWith}.
		\item Для этого придётся раскрыть \t{fib} дважды (как в предыдущих пунктах), а \t{0:fib} "--- один раз.
		\item Раскрыть получится "--- это элементы мы уже считали. Значит, посчитаем их сумму и \t{fib !! 2}.
		\item И так до бесконечности.
		\item Время работы и память неизвестны :)
	\end{itemize}
\end{frame}

\begin{frame}[fragile]{Особенности бесконечных списков}
	\begin{itemize}
		\item У них нет конца.
		\item Если ваша функция пытается дойти до конца списка "--- у неё проблемы:
\begin{minted}{haskell}
getFirstTen xs = take (min 10 (length xs)) xs
length xs  -- Не работает с бесконечными списками.
\end{minted}
		\item Из-за ленивости операции над бесконечными списками реальны:
\begin{minted}{haskell}
take 0 _      = []  -- Тут второй аргумент не вычисляется
take n (x:xs) = x:(take (n - 1) xs)
\end{minted}
		\item Или даже так:
\begin{minted}{haskell}
any' (True :_)  = True
any' (_    :xs) = any' xs
\end{minted}
		\item Функции в домашнем задании должны корректно обрабаывать те бесконечные списки, на которых поведение определено.
		\item Надо писать аккуратно.
	\end{itemize}
\end{frame}

\begin{frame}[fragile]{Упражнение}
	\begin{itemize}
		\item Напишите функцию \t{concat}, который приписывает один список к другому.
		\item Работает в точности как оператор \t{(++)}.
		\item Когда осмысленно получать на вход бесконечный список?
	\end{itemize}
	\pause
\begin{minted}{haskell}
concat (x:xs) ys = x:(concat xs ys)
concat _      ys = ys
\end{minted}
	\begin{itemize}
		\item Если получаем бесконечный список как первый аргумент "--- второй не используется.
		\item Если получаем бесконечный список как второй аргумент "--- он дописывается в конец первого.
		\item В Haskell оператор \t{(++)} действительно работает за линию.
	\end{itemize}
\end{frame}
