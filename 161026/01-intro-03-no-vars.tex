\subsection{Жизнь без переменных}

\begin{frame}{Haskell}
	\begin{itemize}
		\item Haskell "--- чистый функциональный язык программирования.
		\item Компилируемый, статически типизирован, \textit{очень} мощная система типов.
		\item Есть огромное число библиотек.
		\item Очень чистый по сравнению с остальными языками вроде OCaml.
		\item Помимо функциональной чистоты имеет огромное количество интересных особенностей.
		\item На нём действительно можно писать код.
		\item Стандартный компилятор "--- GHC (Glasgow Haskell Compiler).
		\item Рекомендуется использовать в составе \href{https://www.haskell.org/platform/}{Haskell Platform}.
	\end{itemize}
\end{frame}

\begin{frame}{Демонстрация}
	\begin{enumerate}
		\item Интерпретатор \t{ghci} не поддерживает многострочные определения функций.
		\item Поэтому с некоторого момента лучше набирать код в файле, подгружая его в интерпретатор:
			\begin{itemize}
				\item \t{:load file.hs} (\t{:l file.hs}) компилирует \t{file.hs} и подгружает определения в интерпретатор.
				\item \t{:reload} перекомпилирует и переподключит все файлы.
			\end{itemize}
		\item Демо: арифметика, простые типы, сравнения, функции, комментарии, списки, работа со списками, list comprehension.
		\item Упражнение: как найти все Пифагоровы тройки ($x^2 + y^2 = z^2$) при $1 \le x, y, z \le 10$?
		\item Демо: определения функций <<и>>, <<или>>, <<сумма двух чисел>>.
	\end{enumerate}
\end{frame}

\begin{frame}[t]{Жизнь без циклов}
	Упражнения на Python:
	\begin{itemize}
		\item Как посчитать сумму чисел в списке, если нет переменных и циклов?\pause Функция \t{sum}.\pause
		\item Как посчитать сумму квадратов чисел? \pause Определить лямбда-функцию для возведения в квадрат и применить \t{map} с \t{sum}.\pause
		\item Какие вообще операции со списками мы ещё не умеем делать без циклов? \pause В которых элементы влияют друг на друга.
		\item Например: \t{"hello world".split()}.
		\item Тогда функцию \t{split()} пишут <<императивно>>, а всю последующую обработку "--- функционально.
		\item Таким образом императивная сложность изолирована, все кусочки можно тестировать по отдельности.
	\end{itemize}
\end{frame}

\begin{frame}[t,fragile]{Упражнения на рекурсию}
	Факториал:
\begin{minted}{haskell}
fac 0 = 1
fac n = n * fac (n - 1)
\end{minted}
	Степень двойки: \pause
\begin{minted}{haskell}
powerOfTwo 0 = 1
powerOfTwo n = 2 * powerOfTwo (n - 1)
\end{minted}
	Числа Фибоначчи: \pause
\begin{minted}{haskell}
fib n = fib' 0 1 n
-- Одна итерация цикла for
fib' a b 0 = a
fib' a b n = fib' b (a + b) (n - 1)
\end{minted}
	Мы увидели:
	\begin{itemize}
		\item Pattern matching (вместо \t{if} пишем <<шаблон>> для аргумента функции).
		\item Апостроф является корректным символом в названии функции.
	\end{itemize}
\end{frame}

\begin{frame}[t,fragile]{Жизнь без циклов-2}
	Как теперь функцию \t{sum}? \pause
\begin{minted}{haskell}
sum' (x:xs) = x + sum xs
sum' _ = 0
\end{minted}
	\begin{itemize}
		\item Иногда можно делать сложный pattern matching вроде \t{x:xs} (список, первый элемент которого "--- \t{x}, а хвост "--- \t{xs}).
		\item Шаблоны проверяются сверху вниз.
		\item Вместо имени переменной можно написать \t{\_}.
	\end{itemize}
	А как написать в императивном стиле? \pause
\begin{minted}{haskell}
sum' xs = sum'' 0 xs
sum'' a [] = a
sum'' a (x:xs) = sum'' (a + x) xs
\end{minted}
\end{frame}
