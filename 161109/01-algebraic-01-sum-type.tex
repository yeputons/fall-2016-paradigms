\section{Алгебраические типы данных}
\subsection{Тип-произведение}

\begin{frame}
	\tableofcontents[currentsection,currentsubsection]
\end{frame}

\begin{frame}{Алгебры}
	Для понимания этимологии словосочетания <<алгебраический тип>>:
	\begin{itemize}
		\item \textit{Алгебра} "--- это множество $A$, на котором ввели какие-то функции.
		\item Каждая функция принимает некоторое число аргументов из $A$ и возвращает элемент из $A$.
		\item Разные функции могут принимать разное число аргументов.
		\item Никаких дополнительных требований нет.
		\item Рациональные числа являются алгеброй с операциями:
			\begin{enumerate}
				\item $+ \colon \mathrm{Q} \times \mathrm{Q} \to \mathrm{Q}$
				\item $\frac{1}{x} \colon \mathrm{Q} \to \mathrm{Q}$
			\end{enumerate}
		\item Также алгебрами являются все группы:
			\begin{enumerate}
				\item Функция-константа: $e \colon G$
				\item Взятие обратного элемента: $x^{-1} \colon G \to G$
				\item Умножение элементов: $\cdot \colon G \times G \to G$
				\item Дополнительно требуются свойства группы.
			\end{enumerate}
		\item Мораль: если сказали <<введём операцию с элементами>> "--- уже появилась алгебра.
	\end{itemize}
\end{frame}

\begin{frame}{Алгебра типов}
	\begin{itemize}
		\item Типы данных тоже образуют алгебру (обозначим их множество за T):
			\begin{enumerate}
				\item $\t{Int} \colon T$
				\item $\t{Char} \colon T$
				\item Операция <<сделать список>>: $\t{[]} \colon T \to T$
				\item Операция <<сделать кортеж>>: $\t{(,)} \colon T \times T \to T$
			\end{enumerate}
		\item \textit{Алгебраический тип} "--- это тип, собранный из более простых.
		\item Активно используются и в императивных языках: \t{vector<int>}, \t{int[]}, \t{struct \{\}}
		\item \t{struct} имеет специальное название "--- \textit{тип-произведение}
		\item Множество допустимых значений типа-произведения есть декартово произведение множеств допустимых значений составных элементов.
		\item Поэтому операцию <<сделать кортеж>> можно обозначить как умножение типов (так сделано в OCaml).
		\item Пока что ничего нового или полезного.
	\end{itemize}
\end{frame}

\begin{frame}{Упражнение}
	\begin{itemize}
		\item Пусть у интернет-магазина есть три способа оплаты:
			\begin{enumerate}
				\item Банковской картой, нужно знать её данные.
				\item Наличными при получении, ничего дополнительно знать не нужно.
				\item Выставление счёта на QIWI-кошелёк, нужно знать номер телефона.
			\end{enumerate}
		\item Требуется создать тип данных <<способ оплаты>>, который можно хранить и обрабатывать.
		\item Иногда требуется преобразовывать способ оплаты в строку.
		\item Иногда требуется понимать, надо ли что-то делать с сервере для проведения оплаты (если да "--- положить в очередь).
	\end{itemize}
\end{frame}

\begin{frame}[fragile]{Упражнение (C-подход)}
\begin{minted}{cpp}
enum PaymentMethodType { CARD, CASH, QIWI_BILL };
struct PaymentMethod {
  PaymentMethodType type;
  CardInfo card_info;
  char phone[20];
};
\end{minted}
	\begin{itemize}
		\item Надо везде явно смотреть на поле \t{type} и городить if'ы.
		\item Для обработки пишем функции вроде \t{to\_string}, которые разбирают случаи.
		\item Можем случайно обратиться к \t{card\_info}, если не проверим способ оплаты.
		\item Храним больше байт, чем реально надо (можно \t{union}, но там есть свои проблемы).
	\end{itemize}
\end{frame}

\begin{frame}{Упражнение (ООП-подход)}
	\begin{itemize}
		\item Вводим интерфейс \t{PaymentMethod}, а сами методы делаем подклассами.
		\item Общие функции вроде \t{to\_string} вносим в интерфейс.
		\item Специфичные функции либо руками разбирают случаи, либо используют Visitor.
		\item Так обычно и делают.
		\item Можно добавлять как новые классы, так и новые операции с объектами.
	\end{itemize}
\end{frame}

\begin{frame}[fragile]{Упражнение (if'ы)}
\begin{minted}{cpp}
bool need_processing(PaymentMethod *m) {
  if (dynamic_cast<CardPayment*>(m)) {
    return true;
  }
  if (dynamic_cast<CashPayment*>(m)) {
    return false;
  }
  assert(false);
}
\end{minted}
	\begin{itemize}
		\item Можно обрабатывать несколько случаев одинаково.
		\item Узнаем об отсутствующем \t{if} только во время выполнения, если есть тест.
		\item Компилятор не проверит, что мы ничего не забыли.
	\end{itemize}
\end{frame}

\begin{frame}[fragile]{Упражнение (Visitor)}
\begin{minted}{cpp}
bool result;
class NeedProcessingVisitor {
  void accept(CardPayment *p) { result = true; }
  void accept(CashPayment *p) { result = false; }
}
bool need_processign(PaymentMethod *m) {
  m->accept(NeedProcessingVisitor());
  return result;
}
\end{minted}
	\begin{itemize}
		\item Очень много кода не по делу.
		\item Создаётся целый объект в памяти и идут пляски с возвращаемым значением.
	\end{itemize}
\end{frame}

\subsection{Тип-сумма}
\begin{frame}[fragile]{Тип-сумма}
	\begin{itemize}
		\item Можно ввести \textit{тип-сумму}: множество его допустимых значений равно \textit{дизъюнктному объединению}\footnote{объеденение попарно непересекающихся множеств} допустимых значений составных частей.
		\item Чтобы обобщить до суммы произвольных типов, можно каждому значению составной части добавить <<тэг>>.
		\item Пример: тип <<способ оплаты>>:
\begin{minted}{haskell}
data PaymentMethod = BankCard String | Cash | Qiwi String
a = BankCard "1234 5678 9012 3456"
b = Cash
c = Qiwi "+7 812 000 00 00"
\end{minted}
		\item Обычно встречается в функциональных языках.
		\item Именно его наличие обычно подразумевают под <<наличием алгебраических типов данных>>.
	\end{itemize}
\end{frame}

\begin{frame}[fragile]{Тип-сумма: подробности}
\begin{minted}{haskell}
data PaymentMethod = BankCard String | Cash | Qiwi String
\end{minted}
	\begin{itemize}
		\item \t{PaymentMethod} называется \textit{конструктором типа}.
		\item \t{BankCard}, \t{Cash}, \t{Qiwi} называются <<конструкторами данных>>, являются теми самыми <<тэгами>>.
		\item Не путать с конструкторами в ООП!
		\item И конструктор типа, и конструктор данных долнжы начинаться с большой буквы.
		\item Работает с pattern matching:
\begin{minted}{haskell}
to_string (BankCard num) = "BankCard " ++ num
to_string Cash           = "Cash"
to_string (Qiwi phone)   = "Qiwi " ++ phone
\end{minted}
		\item Можно дописать в конец строки с \t{data} слова \t{deriving Show}, чтобы GHCI мог выводить значения типа \t{PaymentMethod}.
	\end{itemize}
\end{frame}
