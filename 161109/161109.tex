\documentclass[utf8,xcolor=table]{beamer}

\usepackage[T2A]{fontenc}
\usepackage[utf8]{inputenc}
\usepackage[english,russian]{babel}
\usepackage{tikz}
\usetikzlibrary{shapes,arrows}
\usepackage{dot2texi}
\usepackage{minted}
\usepackage{ulem}
\usepackage{cmap}
\usepackage{multirow}

\hypersetup{colorlinks,linkcolor=blue,urlcolor=blue}

\mode<presentation>{
	\usetheme{CambridgeUS}
}

\renewcommand{\t}[1]{\ifmmode{\mathtt{#1}}\else{\texttt{#1}}\fi}

\title{Типы в Haskell}
\author{Егор Суворов}
\institute[СПб АУ]{Курс <<Парадигмы и языки программирования>>, подгруппа 3}
\date[09.11.2016]{Среда, 9 ноября  года}

\setlength{\arrayrulewidth}{1pt}

\begin{document}

\begin{frame}
\titlepage
\end{frame}

\begin{frame}{План занятия}
	\tableofcontents
\end{frame}

\section{Алгебраические типы данных}
\subsection{Тип-произведение}

\begin{frame}
	\tableofcontents[currentsection,currentsubsection]
\end{frame}

\begin{frame}{Алгебры}
	Для понимания этимологии словосочетания <<алгебраический тип>>:
	\begin{itemize}
		\item \textit{Алгебра} "--- это множество $A$, на котором ввели какие-то функции.
		\item Каждая функция принимает некоторое число аргументов из $A$ и возвращает элемент из $A$.
		\item Разные функции могут принимать разное число аргументов.
		\item Никаких дополнительных требований нет.
		\item Рациональные числа являются алгеброй с операциями:
			\begin{enumerate}
				\item $+ \colon \mathrm{Q} \times \mathrm{Q} \to \mathrm{Q}$
				\item $\frac{1}{x} \colon \mathrm{Q} \to \mathrm{Q}$
			\end{enumerate}
		\item Также алгебрами являются все группы:
			\begin{enumerate}
				\item Функция-константа: $e \colon G$
				\item Взятие обратного элемента: $x^{-1} \colon G \to G$
				\item Умножение элементов: $\cdot \colon G \times G \to G$
				\item Дополнительно требуются свойства группы.
			\end{enumerate}
		\item Мораль: если сказали <<введём операцию с элементами>> "--- уже появилась алгебра.
	\end{itemize}
\end{frame}

\begin{frame}{Алгебра типов}
	\begin{itemize}
		\item Типы данных тоже образуют алгебру (обозначим их множество за T):
			\begin{enumerate}
				\item $\t{Int} \colon T$
				\item $\t{Char} \colon T$
				\item Операция <<сделать список>>: $\t{[]} \colon T \to T$
				\item Операция <<сделать кортеж>>: $\t{(,)} \colon T \times T \to T$
			\end{enumerate}
		\item \textit{Алгебраический тип} "--- это тип, собранный из более простых.
		\item Активно используются и в императивных языках: \t{vector<int>}, \t{int[]}, \t{struct \{\}}
		\item \t{struct} имеет специальное название "--- \textit{тип-произведение}
		\item Множество допустимых значений типа-произведения есть декартово произведение множеств допустимых значений составных элементов.
		\item Поэтому операцию <<сделать кортеж>> можно обозначить как умножение типов (так сделано в OCaml).
		\item Пока что ничего нового или полезного.
	\end{itemize}
\end{frame}

\begin{frame}{Упражнение}
	\begin{itemize}
		\item Пусть у интернет-магазина есть три способа оплаты:
			\begin{enumerate}
				\item Банковской картой, нужно знать её данные.
				\item Наличными при получении, ничего дополнительно знать не нужно.
				\item Выставление счёта на QIWI-кошелёк, нужно знать номер телефона.
			\end{enumerate}
		\item Требуется создать тип данных <<способ оплаты>>, который можно хранить и обрабатывать.
		\item Иногда требуется преобразовывать способ оплаты в строку.
		\item Иногда требуется понимать, надо ли что-то делать с сервере для проведения оплаты (если да "--- положить в очередь).
	\end{itemize}
\end{frame}

\begin{frame}[fragile]{Упражнение (C-подход)}
\begin{minted}{cpp}
enum PaymentMethodType { CARD, CASH, QIWI_BILL };
struct PaymentMethod {
  PaymentMethodType type;
  CardInfo card_info;
  char phone[20];
};
\end{minted}
	\begin{itemize}
		\item Надо везде явно смотреть на поле \t{type} и городить if'ы.
		\item Для обработки пишем функции вроде \t{to\_string}, которые разбирают случаи.
		\item Можем случайно обратиться к \t{card\_info}, если не проверим способ оплаты.
		\item Храним больше байт, чем реально надо (можно \t{union}, но там есть свои проблемы).
	\end{itemize}
\end{frame}

\begin{frame}{Упражнение (ООП-подход)}
	\begin{itemize}
		\item Вводим интерфейс \t{PaymentMethod}, а сами методы делаем подклассами.
		\item Общие функции вроде \t{to\_string} вносим в интерфейс.
		\item Специфичные функции либо руками разбирают случаи, либо используют Visitor.
		\item Так обычно и делают.
		\item Можно добавлять как новые классы, так и новые операции с объектами.
	\end{itemize}
\end{frame}

\begin{frame}[fragile]{Упражнение (if'ы)}
\begin{minted}{cpp}
bool need_processing(PaymentMethod *m) {
  if (dynamic_cast<CardPayment*>(m)) {
    return true;
  }
  if (dynamic_cast<CashPayment*>(m)) {
    return false;
  }
  assert(false);
}
\end{minted}
	\begin{itemize}
		\item Можно обрабатывать несколько случаев одинаково.
		\item Узнаем об отсутствующем \t{if} только во время выполнения, если есть тест.
		\item Компилятор не проверит, что мы ничего не забыли.
	\end{itemize}
\end{frame}

\begin{frame}[fragile]{Упражнение (Visitor)}
\begin{minted}{cpp}
bool result;
class NeedProcessingVisitor {
  void accept(CardPayment *p) { result = true; }
  void accept(CashPayment *p) { result = false; }
}
bool need_processign(PaymentMethod *m) {
  m->accept(NeedProcessingVisitor());
  return result;
}
\end{minted}
	\begin{itemize}
		\item Очень много кода не по делу.
		\item Создаётся целый объект в памяти и идут пляски с возвращаемым значением.
	\end{itemize}
\end{frame}

\subsection{Тип-сумма}
\begin{frame}[fragile]{Тип-сумма}
	\begin{itemize}
		\item Можно ввести \textit{тип-сумму}: множество его допустимых значений равно \textit{дизъюнктному объединению}\footnote{объединение попарно непересекающихся множеств} допустимых значений составных частей.
		\item Чтобы обобщить до суммы произвольных типов, можно каждому значению составной части добавить <<тэг>>.
		\item Пример: тип <<способ оплаты>>:
\begin{minted}{haskell}
data PaymentMethod = BankCard String | Cash | Qiwi String
a = BankCard "1234 5678 9012 3456"
b = Cash
c = Qiwi "+7 812 000 00 00"
\end{minted}
		\item Обычно встречается в функциональных языках.
		\item Именно его наличие обычно подразумевают под <<наличием алгебраических типов данных>>.
	\end{itemize}
\end{frame}

\begin{frame}[fragile]{Тип-сумма: подробности}
\begin{minted}{haskell}
data PaymentMethod = BankCard String | Cash | Qiwi String
\end{minted}
	\begin{itemize}
		\item \t{PaymentMethod} называется \textit{конструктором типа}.
		\item \t{BankCard}, \t{Cash}, \t{Qiwi} называются <<конструкторами данных>>, являются теми самыми <<тэгами>>.
		\item Не путать с конструкторами в ООП!
		\item И конструктор типа, и конструктор данных долнжы начинаться с большой буквы.
		\item Работает с pattern matching:
\begin{minted}{haskell}
to_string (BankCard num) = "BankCard " ++ num
to_string Cash           = "Cash"
to_string (Qiwi phone)   = "Qiwi " ++ phone
\end{minted}
		\item Можно дописать в конец строки с \t{data} слова \t{deriving Show}, чтобы GHCI мог выводить значения типа \t{PaymentMethod}.
	\end{itemize}
\end{frame}

\subsection{Примеры типов-сумм}
\begin{frame}
	\tableofcontents[currentsection,currentsubsection]
\end{frame}

\begin{frame}[fragile]{CharOrNotFound}
	Поиск элемента по номеру:
\begin{minted}{haskell}
data CharOrNotFound = NotFound | Found Char deriving Show

getItem :: [Char] -> Int -> CharOrNotFound
getItem (x:_ ) 0         = Found x
getItem (x:xs) n | n > 0 = getItem xs (n - 1)
getItem _      _         = NotFound
\end{minted}
	\begin{itemize}
		\item Не требуются <<магические значения>> для ситуации <<элемент не найден>>.
		\item Компилятор проверяют, что мы всегда обрабатываем оба случая.
		\item По типу функции сразу понятно, что она может вернуть.
		\item Нет исключений; функции чистые.
	\end{itemize}	
\end{frame}


\begin{frame}[fragile]{Maybe}
	Можно обобщить до \textit{параметризованного типа}:
\begin{minted}{haskell}
data GetResult a = NotFound | Found a deriving Show

getItem :: [a] -> Int -> GetResult a
getItem (x:_ ) 0         = Found x
getItem (x:xs) n | n > 0 = getItem xs (n - 1)
getItem _ _              = NotFound
\end{minted}
% Показать :t getItem, :t getItem "123" 1
	\begin{itemize}
		\item \t{GetResult} "--- это не тип, это \textit{конструктор типа}.
		\item \t{a} "--- единственный параметр этого конструктора.
		\item А вот \t{GetResult Char} "--- уже конкретный тип:
\begin{minted}{haskell}
data GetResult Char = NotFound | Found Char
\end{minted}
		\item В Haskell такой тип называется \t{Maybe}.
		\item А в Java есть generic-тип \t{Optional<>}).
		\item На самом деле \t{[Int]} "--- это сахар для \t{[] Int}.
	\end{itemize}
\end{frame}

\begin{frame}[t,fragile]{Упражнение}
	\begin{itemize}
		\item Напишите тип для функции \t{getItem}, если бы она использовала \t{Maybe}:
\begin{minted}{haskell}
data Maybe a = Nothing | Just a

getItem :: [a] -> Int -> ???
\end{minted}
		\item Напишите функцию \t{getItem}.
		\item Удалите явное указание типа, проверьте, какой тип вывелcя автоматически (\t{:t getItem} в GHCI).
	\end{itemize}
	\pause
\begin{minted}{haskell}
getItem :: [a] -> Int -> Maybe a
getItem (x:_)  0         = Just x
getItem (x:xs) n | n > 0 = getItem xs (n - 1)
getItem _ _              = Nothing
\end{minted}
\end{frame}

\begin{frame}[fragile]{Either}
	На случай, если хотим сообщить об ошибке:
\begin{minted}{haskell}
data Either a b = Left a | Right b

parseBool :: String -> Either String Bool
parseBool "true"  = Right True
parseBool "false" = Right False
parseBool x       = Left ("Invalid value: " ++ x)
\end{minted}
	\begin{itemize}
		\item Обычно за \t{Right} принимает успешное вычисление (<<правильный>> результат).
		\item А за \t{Left} "--- сообщение об ошибке (<<левый>> результат).
	\end{itemize}
\end{frame}

\begin{frame}[fragile]{Двоичная куча}
\begin{minted}{haskell}
data Heap = Nil | Node Int Heap Heap

Heap 1 (Node 2 (Node 5 (Node 6 Nil Nil) Nil)
               (Node 4 Nil Nil))
       (Node 3 Nil Nil)
\end{minted}
	\begin{center}
		\begin{dot2tex}[scale=0.5,options=-tmath]
			graph G {
			    1 {rank=same 2 3} {rank=same 5 4} { rank=same 6 x }
			    1 -- {2 3};
			    2 -- {5 4};
			    5 -- 6;
			    5 -- x [style=invis];
			    x [style=invis];
			}
		\end{dot2tex}
	\end{center}
\end{frame}

\begin{frame}[fragile]{Min-Max куча}
	Так как вершина всегда хранит либо минимум, либо максимум, это можно указать прямо в типе.
	Тогда точно не запутаемся, где какая вершина, и не надо это явно считать и передавать.
\begin{minted}{haskell}
data MinMaxHeap = Nil | MinNode Int MinMaxHeap 
                      | MaxNode Int MinMaxHeap 
\end{minted}
	Можно даже строже:
\begin{minted}{haskell}
data MinMaxHeap = Nil1 | MinNode Int MaxMinHeap MaxMinHeap
data MaxMinHeap = Nil2 | MaxNode Int MinMaxHeap MinMaxHeap
\end{minted}	
	К сожалению, назвать оба конструктора данных \t{Nil} нельзя.
\end{frame}

\begin{frame}[fragile]{Дерево разбора выражения}
\begin{minted}{haskell}
data BinOp = Add | Sub | Mul | Div deriving Show
data Tree = Number Int
          | Reference String
          | BinaryOperation BinOp Tree Tree
          deriving Show
-- Глубокий pattern matching
fold' Sub (Reference a) (Reference b) | a == b = Number 0
fold' Mul (Reference _) (Number 0)             = Number 0
fold' Mul (Number 0)    (Reference _)          = Number 0
fold' op a b = BinaryOperation op a b
-- Рекурсивное сворачивание выражений
fold (BinaryOperation op a b) =
    fold' op (fold a) (fold b)
fold x = x
\end{minted}
\end{frame}

\begin{frame}[fragile]{Односвязные списки}
\begin{minted}{haskell}
data List a = Empty | Cons a (List a) deriving Show

head' (Cons x _ ) = x
tail' (Cons _ xs) = xs
\end{minted}
	\begin{itemize}
		\item Выше написано почти определение встроенного списка.
		\item \t{[]} "--- это сахар для конструктора \t{Empty}.
		\item \t{:} "--- это сахар для конструктора \t{Cons}.
		\item Конкретно в Haskell любые структуры бывают бесконечными из-за ленивости, не только списки.
		\item Например, бесконечное двоичное дерево имеет право на жизнь.
	\end{itemize}
\end{frame}

\begin{frame}[fragile]{Упражнение}
	\begin{enumerate}
		\item Напишите тип <<двоичное дерево>> (\t{Tree}), в котором у каждой вершины либо 0 детей, либо 2, а каждая вершина содержит значение типа \t{a}.
		\item Напишите функцию \t{tree\_sum}, которая считает сумму в данном ей дереве.
		\item Удалите разбор какого-нибудь случая и запустите GHCI так: \t{ghci~-W~file.hs}
		\item Убедитесь, что выпало предупреждение о неразобранном случае.
		\item Напишите функцию, которая возвращает бесконечное дерево \t{Int}'ов, где каждая вершина содержит номер своего уровня.
		\item Выведите результат на экран, объясните увиденное.
	\end{enumerate}
\end{frame}

\begin{frame}{Промежуточные итоги}
	\begin{itemize}
		\item
			Под <<алгебраическими типами данных>> обычно подразумевают поддержку типов-сумм вместе с типами-произведениями \textit{на уровне языка}.
			Такая поддержка даёт:
			\begin{enumerate}
				\item Более наглядные типы.
				\item Невозможность обратиться к данным из другого <<случая>>.
				\item Pattern matching и сильное упрощение кода.
				\item Предупреждения компилятора о нерассмотренных случаях (ключ \t{-W} для GHC/GHCI).
			\end{enumerate}
		\item Добавлять случаи в тип-сумму обычно после объявления нельзя.
		\item В языках без типов-сумм, но с ООП, обычно используется:
			\begin{itemize}
				\item Наследование от общего предка вместо типов-сумм.
				\item Visitor вместо pattern mactching.
			\end{itemize}
		\item Типы-суммы очень часто возникают при работе с AST.
		\item В Haskell любой пользовательский тип является типом-суммой (возможно, из одного слагаемого).
		\item В Haskell можно параметризовать пользовательские типы.
	\end{itemize}
\end{frame}

\subsection{Замечания из алгебры}
\begin{frame}
	\tableofcontents[currentsection,currentsubsection]
\end{frame}

\begin{frame}[fragile]{Мощь типов-сумм}
	\begin{itemize}
		\item Любой вообразимый тип без стрелок (т.е. без функций) можно представить, как тип-сумму:
\begin{minted}{haskell}
data Bool = True | False
data Int = 0 | 1 | -1 | 2 | -2 | ...
data (Int, Int) = (0,0) | (0,1) | (1,0) | ...
data [Int] = [] | [0] | [0,0] | [1] | ...
\end{minted}
		\item Когда мы пишем параметрирозованный тип, мы на самом деле пишем лишь его <<шаблон>> или функцию, которая возвращает <<реальный>> тип:
			нельзя определить множество значений \t{Maybe a}, не зная множество значений \t{a}.
		\item
			Это называется <<тип высшего порядка>>.
			Этакие шаблоны из C++/generic'и из Java.
		\item
			Отсюда возникло название <<\textit{конструктор типа}>>.
	\end{itemize}
\end{frame}

\begin{frame}[fragile]{Количество значений}
	\begin{itemize}
		\item
			Если заменить каждый тип на количество возможных значений, то названия <<тип-сумма>> и <<тип-произведение>> отображают
			операции, которые надо производить с этими количествами:
\begin{minted}{haskell}
data Bool = True | False  -- Два значения
data Foo = Bool | (Bool, Bool) -- 2 + 2 * 2 = 6 значений
\end{minted}
		\item
			В типах тоже есть дистрибутивность умножения и сложения:
\begin{minted}{haskell}
-- Названия конструктора данных и типа могут совпадать
data Foo1 = Foo1 (Bool, Maybe Int)
data Foo2 = FalseNoInt
          | FalseWithInt Int
          | TrueNoInt
          | TrueWithInt Int
\end{minted}
		\item Enum'ы (перечисления) "--- это типы-суммы, в которых каждое слагаемое имеет ровно одно значение:
\begin{minted}{haskell}
data BinOp = Add | Sub | Mul | Div
\end{minted}
	\end{itemize}
\end{frame}

\subsection{Использование типов-сумм}
\begin{frame}
	\tableofcontents[currentsection,currentsubsection]
\end{frame}

\begin{frame}[t,fragile]{Архиватор}
\only<1-2>{
	\begin{itemize}
		\item Мы пишем консольный архиватор для одного файла (вроде gzip).
		\item Хотим написать тип, который хранит параметры командной строки: входной файл, выходной файл, сжатие/расжатие.
	\end{itemize}
}

\begin{onlyenv}<1-2>
Попытка 1:
\begin{minted}{haskell}
data Operation = Compress | Decompress deriving Show
data Args = Args { inp :: String
                 , out :: String
                 , op  :: Operation
                 } deriving Show
\end{minted}
Замечание: тут у нас тип-сумма с одним-конструктором, простых типов-произведений в Haskell нет.
\end{onlyenv}
\only<2>{А теперь хотим добавить пароль для запаковки/распаковки (нужен не всегда)}

\begin{onlyenv}<3-4>
Попытка 2:
\begin{minted}{haskell}
data Operation = Compress | Decompress deriving Show
data Args = Args { inp :: String
                 , out :: String
                 , op  :: Operation
                 , pwd :: Maybe String
                 } deriving Show
\end{minted}
\end{onlyenv}
\only<4>{Новые параметры: уровень логирования (0-3), флаг <<не перезаписывать файл при создании>>}

\begin{onlyenv}<5-7>
Попытка 3:
\begin{minted}{haskell}
data Operation = Compress | Decompress deriving Show
data Args = Args { inp :: String
                 , out :: String
                 , op  :: Operation
                 , pwd :: Maybe String
                 , logLevel  :: Int
                 , overwrite :: Bool
                 } deriving Show
\end{minted}
\end{onlyenv}
\only<6-7>{
\begin{itemize}
	\item Новые параметры: формат архива, степень сжатия.
	\only<7>{
	\item Если просто добавить поле \t{Maybe (Format, Int)}, то появится инвариант и структура может стать некорректной.
	\item Можно убрать \t{op} и заменить его на новый \t{Maybe}.
	}
\end{itemize}
}

\begin{onlyenv}<8-9>
Попытка 4:
\begin{minted}{haskell}
-- data Operation = Compress | Decompress deriving Show
data Args = Args { inp  :: String
                 , out :: String
                 , comprArgs :: Maybe (Format, Int)
                 , pwd :: Maybe String
                 , logLevel  :: Int
                 , overwrite :: Bool
                 } deriving Show
\end{minted}
\end{onlyenv}
\only<9>{
\begin{itemize}
	\item А если теперь добавим третью операцию, то всё сломается.
	\item А если она при этом будет только читать архив (но никуда не писать), всё сломается ещё раз.
	\item При взгляде на структуру неочевидно, как узнать, сжимаем или разжимаем, хотя вся информация в ней есть.
\end{itemize}
}

\begin{onlyenv}<10>
Намного лучше разделить параметры на совсем общие и относящиеся к операции.

Попытка 5, финальная:
\begin{minted}{haskell}
data CommonArgs = Common { logLevel  :: Int }
data OperationArgs = -- Заодно имена лучше отражают суть.
    Compress   { input     :: String, archive :: String,
                 pwd       :: Maybe String,
                 overwrite :: Bool,
                 format    :: Format, level :: Int }
  | Decompress { archive   :: String, output :: String,
                 pwd       :: Maybe String,
                 overwrite :: Bool }
data Args = Args (CommonArgs, OperationArgs)
\end{minted}
Мораль: лучше думать в терминах <<какие по смыслу бывают ситуации>>, чем
<<как хранить всё одновременно>>
\end{onlyenv}
\end{frame}

\begin{frame}[t,fragile]{Хранение URL}
	URL-адреса бывают:
	\begin{itemize}
		\item Относительные: \t{../images/facepalm.jpg}.
		\item Абсолютные, бывают:
			\begin{itemize}
				\item На том же домене: \t{sewiki/index.php}.
				\item На другом домене, причём:
					\begin{itemize}
						\item Та же схема (протокол): \t{google.com/humans.txt}
	                    \item Другая схема: \t{ftp://mirror.yandex.ru/}
                   	\end{itemize}
			\end{itemize}
	\end{itemize}
\begin{onlyenv}<1>
	Можно закодировать так\footnote{True story: раздел <<Thinking in Sum Types>> по \href{https://chadaustin.me/2015/07/sum-types/}{ссылке}}:
\begin{minted}{haskell}
data URL = URL (Maybe (Maybe (Maybe String, String))) String deriving Show
URL Nothing        "../images/facepalm.jpg"
URL (Just Nothing) "sewiki/index.php"
URL (Just (Just (Nothing   , "google.com"      ))) "humans.txt"
URL (Just (Just (Just "ftp", "mirror.yandex.ru"))) ""
\end{minted}
Ужасно, не правда ли?
\end{onlyenv}

\begin{onlyenv}<2>
А можно так:
\begin{minted}{haskell}
data URL = Relative String
         | Absolute String
         | OtherDomain { domain :: String, path :: String }
         | FullUrl     { schema :: String,
                         domain :: String, path :: String }
\end{minted}
Мораль: иногда может помочь <<раскрыть по дистрибутивности>>.
\end{onlyenv}
\end{frame}

\begin{frame}[t,fragile]{Физика}
\begin{minted}{haskell}
data Time = Sec  Double | Min  Double
          | Msec Double | Usec Double
          deriving Show
diff (Sec x) (Sec y) = Sec (x - y)
diff x y = diff (to_sec x) (to_sec y)

diff' x y = Sec (x' - y')
  where (Sec x') = to_sec x -- pattern matching
        (Sec y') = to_sec y
\end{minted}
	\begin{onlyenv}<1>
\begin{minted}{haskell}
to_sec (Sec x) = Sec x
to_sec (Msec x) = Sec (x / 1000)
to_sec (Usec x) = Sec (x / 1000000)
to_sec (Min x) = Sec (x * 60)
\end{minted}
	\end{onlyenv}
	\only<2>{
	\begin{itemize}
		\item Компилятор обяжет вас указать единицы измерения во всех местах.
		\item Все проверки будут сделаны во время компиляции.
		\item Можно работать в удобных единицах и автоматически конвертировать из остальных.
	\end{itemize}
	}
\end{frame}

\begin{frame}[fragile]{Резюме-1}
	\begin{itemize}
		\item Типы-суммы очень естественны, если про них думать.
		\item Полезны, когда возможные значения типа разбиваются на фиксированное число групп, которые сильно отличаются.
		\item Если число групп неизвестно "--- поможет ООП, наследование и Visitor.
		\item Если группы одинаковы "--- возможно, имеет смысл сделать тип-сумму внутри, а не снаружи:
\begin{minted}{haskell}
data Tree = Sum Tree Tree | Mul Tree Tree
-- против
data BinOp = Sum | Mul | Sub
data Tree = BinaryOperation BinOp Tree Tree
\end{minted}
		\item В императивных языках типов-сумм обычно нет, используем наследование и Visitor.
	\end{itemize}
\end{frame}

\begin{frame}{Резюме-2}
	\begin{itemize}
		\item
			Типы-суммы позволяют выражать многие инварианты данных (<<флаг степени сжатия есть только при сжатии файла>>) на уровне типа.
		\item
			Компилятор может проверять сохранение этих инвариантов.
		\item
			Компилятор может проверить то, что все случаи везде разобраны.
		\item
			При поддержке типов-сумм на уровне языка код получается намного короче и чище, чем через ООП.
		\item
			Pattern matching + алгебраические типы = мощь.
	\end{itemize}
\end{frame}

\section{Классы типов}
\subsection{Что и зачем}

\begin{frame}
	\tableofcontents[currentsection,currentsubsection]
\end{frame}

\begin{frame}[fragile]{Pattern Matching и \t{==}}
\begin{minted}{haskell}
data IntList = Empty | Cons Int IntList deriving Show
a = Cons 1 (Cons 2 Empty)
b = Cons 1 (Cons 2 Empty)
c = Cons 1 (Cons 3 Empty)

isA :: IntList -> Bool
isA (Cons 1 (Cons 2 Empty)) = True
isA _                       = False

isA a -- True
isA b -- True
isA c -- False
a == b  -- ошибка компиляции?
a == c  -- ошибка компиляции?
\end{minted}
\end{frame}

\begin{frame}[fragile]{Eq}
	\begin{itemize}
		\item Pattern Matching "--- конструкция на уровне языка.
		\item \t{==} "--- просто некоторая функция с таким названием.
		\item В C++ мы бы написали перегрузку функции/оператора.
		\item В Haskell пишем так:
\begin{minted}{haskell}
instance Eq IntList where
  Empty       == Empty       = True
  (Cons x xs) == (Cons y ys) = (x == y) && (xs == ys)
  _           == _           = False

a == b  -- True
a == c  -- False
b == c  -- False
Empty == Empty           -- True
Empty /= (Cons 1 Empty)  -- True, /= тоже работает
\end{minted}
	\end{itemize}
\end{frame}

\begin{frame}[fragile]{class Eq}
	\begin{itemize}
		\item \t{Eq} "--- это \textit{класс типов}, который описывает, что к типам можно применять определённые функции:
\begin{minted}{haskell}
class Eq a where
  (==) :: a -> a -> Bool
  (/=) :: a -> a -> Bool
\end{minted}
		\item Говорим, что тип \t{a} лежит в классе \t{Eq} тогда и только тогда, когда для него есть функции \t{(==)} и \t{(/=)}
		\item Класс типов "--- это такой <<интерфейс>> для типов.
		\item Некоторые функции требуют, чтобы параметры были в определённых классах:
\begin{minted}{haskell}
lookup :: Eq a => a -> [(a, b)] -> Maybe b
\end{minted}
		\item Слово \t{instance} на предыдущем слайде добавляло \t{IntList} в класс \t{Eq}.
		\item Не путать с классами объектов из ООП!
	\end{itemize}
\end{frame}

\begin{frame}[fragile]{Реализации по умолчанию}
	\begin{itemize}
		\item Для \t{IntList} мы реализовали только \t{==}, а \t{/=} получили автоматом.
		\item В классе можно указывать реализацию по умолчанию:
\begin{minted}{haskell}
class Eq a where
  (==) :: a -> a -> Bool
  (==) a b = not (a /= b)
  (/=) :: a -> a -> Bool
  (/=) a b = not (a == b)
\end{minted}
		\item Тогда \textit{минимальное полное определение} для \t{Eq} "--- это либо \t{==}, либо \t{/=}.
	\end{itemize}
\end{frame}

\subsection{Для параметризованных типов}
\begin{frame}[fragile]{Класс Eq для списков}
	\begin{itemize}
		\item Пусть есть свой класс для списков:
\begin{minted}{haskell}
data List a = Empty | Cons a (List a)
\end{minted}
		\item Разумно считать, что списки равны, если равны элементы:
\begin{minted}{haskell}
instance Eq (List a) where
  Empty       == Empty       = True
  (Cons x xs) == (Cons y ys) = (x == y) && (xs == ys)
  _ == _                     = False
\end{minted}
		\item Не скомпилируется, потому что элементы произвольного типа \t{a} нельзя сранивать.
		\item Надо добавить \textit{контекст} "--- сказать, что списки можно сравнивать только если можно сранивать элементы:
\begin{minted}{haskell}
instance Eq a => Eq (List a) where
\end{minted}
	\end{itemize}
\end{frame}

\begin{frame}[fragile]{Классы для структур данных}
	\begin{itemize}
		\item Иногда хочется указать, что структура данных обладает некоторым свойством:
\begin{minted}{haskell}
class Mappable f where
  map' :: (a -> b) -> f a -> f b
\end{minted}
		\item \t{Mappable} "--- что-то, содержащее элементы произвольного типа, к чему можно делать \t{map'}.
		\item Можно реализовать:
\begin{minted}{haskell}
instance Mappable List where
  map' _ Empty = Empty
  map' f (Cons x xs) = Cons (f x) (map' f xs)

map' (+1) (Cons 1 (Cons 10 Empty))
-- Cons 2 (Cons 11 Empty)
\end{minted}
	\end{itemize}
\end{frame}

\begin{frame}[fragile]{Functor}
	\begin{itemize}
		\item Аналог \t{Mappable} в Haskell называется \t{Functor}, в нём определена функция \t{fmap}.
		\item \t{fmap} должна удовлетворять некоторым аксиомам, но их компилятор сам не проверит:
\begin{minted}{haskell}
fmap id x == x
fmap (f . g) x == fmap f (fmap g x)
\end{minted}
		\item После реализации \t{Functor} наш новый тип можно использовать в том числе в старых функциях:
\begin{minted}{haskell}
a = (Cons 1 (Cons 2 (Cons 3 Empty)))
fmap (+1) [1,2,3]
fmap (+1) a
-- Встроенный оператор замены значений на константу
10 <$ [1,2,3]  -- [10,10,10]
10 <$ a
\end{minted}
	\end{itemize}
\end{frame}

\subsection{Прочие плюшки}
\begin{frame}[fragile]{Свои классы типов}
	Можно создать свой собственный класс:
\begin{minted}{haskell}
class HasSize a where
  size :: a -> Int

instance HasSize [a] where
  size xs = length xs

data Tree a = Nil | Node a (Tree a) (Tree a)
instance HasSize (Tree a) where
  size Nil = 0
  size (Node _ l r) = size l + size r
\end{minted}
	\begin{itemize}
		\item
			Можно определить даже для старых типов.
			В Java/C++ такое можно сделать только внешними перегруженными функциями.
		\item Новые функции можно использовать где угодно с любыми типами.
	\end{itemize}
\end{frame}

\begin{frame}[fragile]{Стандартные классы}
	\begin{itemize}
		\item \t{Show} "--- то, что можно вывести на экран.
		\item \t{Eq} "--- операторы \t{==} и \t{/=}.
		\item \t{Ord} "--- операторы \t{<}, \t{<=} и прочие.
		\item \t{Functor} "--- структура данных, на которой есть \t{map}.
		\item \t{Foldable} "--- структура данных, на которой есть \t{foldr} (по сути, умеет разворачиваться в список).
		\item Для первых трёх Haskell умеет сам генерировать адекватные реализации, если попросить:
\begin{minted}{haskell}
data List a = Empty | Cons a (List a)
            deriving (Show, Eq, Ord)
\end{minted}
		\item Порядок <<лексикографический>> (более ранний конструктор меньше).
	\end{itemize}
\end{frame}

\begin{frame}[fragile]{Зависимости классов}
	\begin{itemize}
		\item Как определить \t{Ord}? Хотим, чтобы тип класса \t{Ord} автоматически подходил везде, где нужен лишь \t{Eq}.
		\item Можно скопировать \t{==} и \t{/=} в \t{Ord}, но тогда:
			\begin{enumerate}
				\item Можно случайно реализовать и \t{Ord}, и \t{Eq}, причём по-разному.
				\item Получаем <<утиную типизацию>>: один класс вкладывается в другой, если есть функции с такими же названиями.
				\item Это нехорошо: придётся следить, что названия функций вообще нигде не пересекаются.
			\end{enumerate}
		\item Можно сказать, что \t{Ord} определяет только новые операторы, но тогда мы можем определить тип с \t{<}, но без \t{==}, что странно.
		\item Решение: класс \t{Ord} требует, чтобы тип также был в классе \t{Eq}:
\begin{minted}{haskell}
class Eq a => Ord a where
  (<), (<=), (>=), (>)  :: a -> a -> Bool
\end{minted}
		\item Если где-то пишем контекст \t{Ord a}, то \t{Eq a} появляется неявно.
		\item В частности, в реализациях по умолчанию в \t{Ord a} можно использовать \t{==} и \t{/=}.
	\end{itemize}
\end{frame}

\begin{frame}[fragile]{Автовывод типов}
\begin{minted}{haskell}
-- Ord a => a -> a -> a
max' a b = a == b

-- (Functor f, Eq a) => a -> f a -> f (Maybe a)
removeByValue x ys = fmap f ys
  where
    f y | x == y    = Nothing
        | otherwise = Just y
\end{minted}
	Если в файле не видно разных функций с одинаковым названием из разных классов, то компилятор может автоматически вывести ограничения на типы (контекст).
\end{frame}

\begin{frame}[fragile]{Резюме}
	\begin{itemize}
		\item Альтернатива классам типов "--- интерфейсы из ООП или перегрузки функций.
		\item Перегрузки функций не отражают связи между разными функциями (вроде \t{==} и \t{/=}).
		\item Интерфейсы из ООП \textit{обычно} надо определять в момент создания каждого типа (не добавить интерфейс к уже существующему).
		\item Интерфейсы из ООП \textit{обычно} не позволяют делать реализации по умолчанию "--- надо писать руками.
		\item Классы типов всё это позволяют.
		\item Компилятор умеет автоматически выводить нужный контекст.
		\item В Haskell типы классов используется везде, где есть хотя бы доля обобщаемости.
	\end{itemize}
\end{frame}

\begin{frame}[fragile]{Упражнение}
	\begin{enumerate}
		\item Пусть имеется следующее дерево со значениями в листьях и произвольным числом детей:
\begin{minted}{haskell}
data Tree a = Leaf a | Node [a] deriving Show
\end{minted}
		\item Добавьте Tree в класс \t{Eq}. Подсказка: списки там уже есть.
		\item Пусть есть такой класс для структур, в которых можно развернуть порядок элементов:
\begin{minted}{haskell}
class Reversible f where
   reverse :: f a -> f a
\end{minted}
		\item Добавьте списки в этот класс:
\begin{minted}{haskell}
class Reversible [] where
\end{minted}
		\item Добавьте дерево в этот класс:
\begin{minted}{haskell}
class Reversible Tree where
\end{minted}
	\end{enumerate}
\end{frame}

\subsection{Прочие плюшки}
\begin{frame}[fragile]{Свои классы типов}
	Можно создать свой собственный класс:
\begin{minted}{haskell}
class HasSize a where
  size :: a -> Int

instance HasSize [a] where
  size xs = length xs

data Tree a = Nil | Node a (Tree a) (Tree a)
instance HasSize (Tree a) where
  size Nil = 0
  size (Node _ l r) = size l + size r
\end{minted}
	\begin{itemize}
		\item
			Можно определить даже для старых типов.
			В Java/C++ такое можно сделать только внешними перегруженными функциями.
		\item Новые функции можно использовать где угодно с любыми типами.
	\end{itemize}
\end{frame}

\begin{frame}[fragile]{Стандартные классы}
	\begin{itemize}
		\item \t{Show} "--- то, что можно вывести на экран.
		\item \t{Eq} "--- операторы \t{==} и \t{/=}.
		\item \t{Ord} "--- операторы \t{<}, \t{<=} и прочие.
		\item \t{Functor} "--- структура данных, на которой есть \t{map}.
		\item \t{Foldable} "--- структура данных, на которой есть \t{foldr} (по сути, умеет разворачиваться в список).
		\item Для первых трёх Haskell умеет сам генерировать адекватные реализации, если попросить:
\begin{minted}{haskell}
data List a = Empty | Cons a (List a)
            deriving (Show, Eq, Ord)
\end{minted}
		\item Порядок <<лексикографический>> (более ранний конструктор меньше).
	\end{itemize}
\end{frame}

\begin{frame}[fragile]{Зависимости классов}
	\begin{itemize}
		\item Как определить \t{Ord}? Хотим, чтобы тип класса \t{Ord} автоматически подходил везде, где нужен лишь \t{Eq}.
		\item Можно скопировать \t{==} и \t{/=} в \t{Ord}, но тогда:
			\begin{enumerate}
				\item Можно случайно реализовать и \t{Ord}, и \t{Eq}, причём по-разному.
				\item Получаем <<утиную типизацию>>: один класс вкладывается в другой, если есть функции с такими же названиями.
				\item Это нехорошо: придётся следить, что названия функций вообще нигде не пересекаются.
			\end{enumerate}
		\item Можно сказать, что \t{Ord} определяет только новые операторы, но тогда мы можем определить тип с \t{<}, но без \t{==}, что странно.
		\item Решение: класс \t{Ord} требует, чтобы тип также был в классе \t{Eq}:
\begin{minted}{haskell}
class Eq a => Ord a where
  (<), (<=), (>=), (>)  :: a -> a -> Bool
\end{minted}
		\item Если где-то пишем контекст \t{Ord a}, то \t{Eq a} появляется неявно.
		\item В частности, в реализациях по умолчанию в \t{Ord a} можно использовать \t{==} и \t{/=}.
	\end{itemize}
\end{frame}

\begin{frame}[fragile]{Автовывод типов}
\begin{minted}{haskell}
-- Ord a => a -> a -> a
max' a b = a == b

-- (Functor f, Eq a) => a -> f a -> f (Maybe a)
removeByValue x ys = fmap f ys
  where
    f y | x == y    = Nothing
        | otherwise = Just y
\end{minted}
	Если в файле не видно разных функций с одинаковым названием из разных классов, то компилятор может автоматически вывести ограничения на типы (контекст).
\end{frame}

\begin{frame}[fragile]{Резюме}
	\begin{itemize}
		\item Альтернатива классам типов "--- интерфейсы из ООП или перегрузки функций.
		\item Перегрузки функций не отражают связи между разными функциями (вроде \t{==} и \t{/=}).
		\item Интерфейсы из ООП \textit{обычно} надо определять в момент создания каждого типа (не добавить интерфейс к уже существующему).
		\item Интерфейсы из ООП \textit{обычно} не позволяют делать реализации по умолчанию "--- надо писать руками.
		\item Классы типов всё это позволяют.
		\item Компилятор умеет автоматически выводить нужный контекст.
		\item В Haskell типы классов используется везде, где есть хотя бы доля обобщаемости.
	\end{itemize}
\end{frame}

\begin{frame}[fragile]{Упражнение}
	\begin{enumerate}
		\item Пусть имеется следующее дерево со значениями в листьях и произвольным числом детей:
\begin{minted}{haskell}
data Tree a = Leaf a | Node [a] deriving Show
\end{minted}
		\item Добавьте Tree в класс \t{Eq}. Подсказка: списки там уже есть.
		\item Пусть есть такой класс для структур, в которых можно развернуть порядок элементов:
\begin{minted}{haskell}
class Reversible f where
   reverse :: f a -> f a
\end{minted}
		\item Добавьте списки в этот класс:
\begin{minted}{haskell}
class Reversible [] where
\end{minted}
		\item Добавьте дерево в этот класс:
\begin{minted}{haskell}
class Reversible Tree where
\end{minted}
	\end{enumerate}
\end{frame}


\end{document}
