\subsection{Использование типов-сумм}
\begin{frame}
	\tableofcontents[currentsection,currentsubsection]
\end{frame}

\begin{frame}[t,fragile]{Архиватор}
\only<1-2>{
	\begin{itemize}
		\item Мы пишем консольный архиватор для одного файла (вроде gzip).
		\item Хотим написать тип, который хранит параметры командной строки: входной файл, выходной файл, сжатие/расжатие.
	\end{itemize}
}

\begin{onlyenv}<1-2>
Попытка 1:
\begin{minted}{haskell}
data Operation = Compress | Decompress deriving Show
data Args = { in  :: String
            , out :: String
            , op  :: Operation
            } deriving Show
\end{minted}
\end{onlyenv}
\only<2>{А теперь хотим добавить пароль для запаковки/распаковки (нужен не всегда)}

\begin{onlyenv}<3-4>
Попытка 2:
\begin{minted}{haskell}
data Operation = Compress | Decompress deriving Show
data Args = { in  :: String
            , out :: String
            , op  :: Operation
            , pwd :: Maybe String
            } deriving Show
\end{minted}
\end{onlyenv}
\only<4>{Новые параметры: уровень логирования (0-3), флаг <<не перезаписывать файл при создании>>}

\begin{onlyenv}<5-7>
Попытка 3:
\begin{minted}{haskell}
data Operation = Compress | Decompress deriving Show
data Args = { in  :: String
            , out :: String
            , op  :: Operation
            , pwd :: Maybe String
            , logLevel  :: Int
            , overwrite :: Bool
            } deriving Show
\end{minted}
\end{onlyenv}
\only<6-7>{
\begin{itemize}
	\item Новые параметры: формат архива, степень сжатия.
	\only<7>{
	\item Если просто добавить поле \t{Maybe (Format, Int)}, то появится инвариант и структура может стать некорректной.
	\item Можно убрать \t{op} и заменить его на новый \t{Maybe}.
	}
\end{itemize}
}

\begin{onlyenv}<8-9>
Попытка 4:
\begin{minted}{haskell}
-- data Operation = Compress | Decompress deriving Show
data Args = { in  :: String
            , out :: String
            , comprArgs :: Maybe (Format, Int)
            , pwd :: Maybe String
            , logLevel  :: Int
            , overwrite :: Bool
            } deriving Show
\end{minted}
\end{onlyenv}
\only<9>{
\begin{itemize}
	\item А если теперь добавим третью операцию, то всё сломается.
	\item А если она при этом будет только читать архив (но никуда не писать), всё сломается ещё раз.
	\item При взгляде на структуру неочевидно, как узнать, сжимаем или разжимаем, хотя вся информация в ней есть.
\end{itemize}
}

\begin{onlyenv}<10>
Намного лучше разделить параметры на совсем общие и относящиеся к операции.

Попытка 5, финальная:
\begin{minted}{haskell}
data CommonArgs = Common { logLevel  :: Int }
data OperationArgs = -- Заодно имена лучше отражают суть.
    Compress   { input     :: String, archive :: String,
                 pwd       :: Maybe String,
                 overwrite :: Bool,
                 format    :: Format, level :: Int }
  | Decompress { archive   :: String, output :: String,
                 pwd       :: Maybe String,
                 overwrite :: Bool }
data Args = Args (CommonArgs, OperationArgs)
\end{minted}
Мораль: лучше думать в терминах <<какие по смыслу бывают ситуации>>, чем
<<как хранить всё одновременно>>
\end{onlyenv}
\end{frame}

\begin{frame}[t,fragile]{Хранение URL}
	URL-адреса бывают:
	\begin{itemize}
		\item Относительные: \t{../images/facepalm.jpg}.
		\item Абсолютные, бывают:
			\begin{itemize}
				\item На том же домене: \t{sewiki/index.php}.
				\item На другом домене, причём:
					\begin{itemize}
						\item Та же схема (протокол): \t{google.com/humans.txt}
	                    \item Другая схема: \t{ftp://mirror.yandex.ru/}
                   	\end{itemize}
			\end{itemize}
	\end{itemize}
\begin{onlyenv}<1>
	Можно закодировать так\footnote{True story: раздел <<Thinking in Sum Types>> по \href{https://chadaustin.me/2015/07/sum-types/}{ссылке}}:
\begin{minted}{haskell}
data URL = URL (Maybe (Maybe (Maybe String, String)), String)
(Nothing     , "../images/facepalm.jpg")
(Just Nothing, "sewiki/index.php")
(Just (Just (Nothing   , "google.com"      )), "humans.txt")
(Just (Just (Just "ftp", "mirror.yandex.ru")), "")
\end{minted}
Ужасно, не правда ли?
\end{onlyenv}

\begin{onlyenv}<2>
А можно так:
\begin{minted}{haskell}
data URL = Relative String
         | Absolute String
         | OtherDomain { domain :: String, path :: String }
         | FullUrl     { schema :: string,
                         domain :: String, path :: String }
\end{minted}
Мораль: иногда может помочь <<раскрыть по дистрибутивности>>.
\end{onlyenv}
\end{frame}

\begin{frame}[t,fragile]{Физика}
\begin{minted}{haskell}
data Time = Sec Double
          | Msec Double
          | Usec Double
          | Min Double
get_days (Sec x) = x / (24 * 60 * 60)
get_days y = get_days (to_sec y)

get_days' x = x_sec / (24 * 60 * 60)
  where (Sec x_sec) = to_sec x -- pattern matching
\end{minted}
\begin{onlyenv}<1>
\begin{minted}{haskell}
to_sec (Sec x) = Sec x
to_sec (Msec x) = Sec (x / 1000)
to_sec (Usec x) = Sec (x / 1000000)
to_sec (Min x) = Sec (x * 60)
\end{minted}
\end{onlyenv}
\only<2>{
	\begin{itemize}
		\item Компилятор обяжет вас указать единицы измерения во всех местах.
		\item Все проверки будут сделаны во время компиляции.
		\item Можно работать в удобных единицах и автоматически конвертировать из остальных.
	\end{itemize}
}
\end{frame}

\begin{frame}[fragile]{Резюме-1}
	\begin{itemize}
		\item Типы-суммы очень естественны, если про них думать.
		\item Полезны, когда возможные значения типа разбиваются на фиксированное число групп, которые сильно отличаются.
		\item Если число групп неизвестно "--- поможет ООП, наследование и Visitor.
		\item Если группы одинаковы "--- возможно, имеет смысл сделать тип-сумму внутри, а не снаружи:
\begin{minted}{haskell}
data BinaryOperation = Sum Tree Tree | Mul Tree Tree
-- против
data BinOp = Sum | Mul | Sub
data BinaryOperation = BinaryOperation BinOp Tree Tree
\end{minted}
		\item В императивных языках типов-сумм обычно нет, используем наследование и Visitor.
	\end{itemize}
\end{frame}

\begin{frame}{Резюме-2}
	\begin{itemize}
		\item
			Типы-суммы позволяют выражать многие инварианты данных (<<флаг степени сжатия есть только при сжатии файла>>) на уровне типа.
		\item
			Компилятор может проверять сохранение этих инвариантов.
		\item
			Компилятор может проверить то, что все случаи везде разобраны.
		\item
			При поддержке типов-сумм на уровне языка код получается намного короче и чище, чем через ООП.
		\item
			Pattern matching + алгебраические типы = мощь.
	\end{itemize}
\end{frame}
