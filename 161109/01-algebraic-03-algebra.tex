\subsection{Замечания из алгебры}
\begin{frame}
	\tableofcontents[currentsection,currentsubsection]
\end{frame}

\begin{frame}[fragile]{Мощь типов-сумм}
	\begin{itemize}
		\item Любой вообразимый тип без стрелок (т.е. без функций) можно представить, как тип-сумму:
\begin{minted}{haskell}
data Bool = True | False
data Int = 0 | 1 | -1 | 2 | -2 | ...
data (Int, Int) = (0,0) | (0,1) | (1,0) | ...
data [Int] = [] | [0] | [0,0] | [1] | ...
\end{minted}
		\item Когда мы пишем параметрирозованный тип, мы на самом деле пишем лишь его <<шаблон>> или функцию, которая возвращает <<реальный>> тип:
			нельзя определить множество значений \t{Maybe a}, не зная множество значений \t{a}.
		\item
			Это называется <<тип высшего порядка>>.
			Этакие шаблоны из C++/generic'и из Java.
		\item
			Напоминание: отсюда возникло название <<\textit{конструктор типа}>>.
	\end{itemize}
\end{frame}

\begin{frame}[fragile]{Количество значений}
	\begin{itemize}
		\item
			Если заменить каждый тип на количество возможных значений, то названия <<тип-сумма>> и <<тип-произведение>> отображают
			операции, которые надо производить с этими количествами:
\begin{minted}{haskell}
data Bool = True | False  -- Два значения
data Foo = Bool | (Bool, Bool) -- 2 + 2 * 2 = 6 значений
\end{minted}
		\item
			В типах тоже есть дистрибутивность умножения и сложения:
\begin{minted}{haskell}
-- Названия конструктора данных и типа могут совпадать
data Foo1 = Foo1 (Bool, Maybe Int)
data Foo2 = FalseNoInt
          | FalseWithInt Int
          | TrueNoInt
          | TrueWithInt Int
\end{minted}
		\item Enum'ы (перечисления) "--- это типы-суммы, в которых каждое слагаемое имеет ровно одно значение:
\begin{minted}{haskell}
data BinOp = Add | Sub | Mul | Div
\end{minted}
	\end{itemize}
\end{frame}

\begin{frame}{Алгебры}
	Для понимания этимологии словосочетания <<алгебраический тип>>:
	\begin{itemize}
		\item \textit{Алгебра} "--- это множество $A$, на котором ввели какие-то функции.
		\item Каждая функция принимает некоторое число аргументов из $A$ и возвращает элемент из $A$.
		\item Разные функции могут принимать разное число аргументов.
		\item Никаких дополнительных требований нет.
		\item Рациональные числа являются алгеброй с операциями:
			\begin{enumerate}
				\item $+ \colon \mathrm{Q} \times \mathrm{Q} \to \mathrm{Q}$
				\item $- \colon \mathrm{Q} \to \mathrm{Q}$ (унарный минус)
			\end{enumerate}
		\item Также алгебрами являются все группы:
			\begin{enumerate}
				\item Функция-константа: $e \colon G$
				\item Взятие обратного элемента: $x^{-1} \colon G \to G$
				\item Умножение элементов: $\cdot \colon G \times G \to G$
				\item Дополнительно требуются свойства группы.
			\end{enumerate}
		\item Мораль: если сказали <<введём операцию с элементами>> "--- уже появилась алгебра.
	\end{itemize}
\end{frame}

\begin{frame}{Алгебра типов}
	\begin{itemize}
		\item Типы данных тоже образуют алгебру (обозначим их множество за T):
			\begin{enumerate}
				\item $\t{Int} \colon T$
				\item $\t{Char} \colon T$
				\item Операция <<сделать список>>: $\t{[]} \colon T \to T$
				\item Операция <<сделать кортеж (тип-произведение)>>: $\t{(,)} \colon T \times T \to T$
				\item Операция <<тип-сумма>>: $\t{|} \colon T \times T \to T$
			\end{enumerate}
		\item \textit{Алгебраический тип} "--- это тип, собранный из более простых.
		\item Активно используются и в императивных языках: \t{vector<int>}, \t{int[]}, \t{struct \{\}}
		\item Операцию <<сделать кортеж>> можно обозначить как умножение типов (так сделано в OCaml).
		\item В императивных языках алгебра типов обычно более скудная.
		\item
			Типы высшего порядка (параметризованные) "--- это на самом деле функции над типами,
			то есть в алгебру в Haskell входят не только типы; подробностей не будет.
	\end{itemize}
\end{frame}
