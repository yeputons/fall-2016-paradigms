\section{Классы типов}
\subsection{Что и зачем}

\begin{frame}
	\tableofcontents[currentsection,currentsubsection]
\end{frame}

\begin{frame}[fragile]{Pattern Matching и \t{==}}
\begin{minted}{haskell}
data IntList = Empty | Cons Int IntList deriving Show
a = Cons 1 (Cons 2 Empty)
b = Cons 1 (Cons 2 Empty)
c = Cons 1 (Cons 3 Empty)

isA :: IntList -> Bool
isA (Cons 1 (Cons 2 Empty)) = True
isA _                       = False

isA a -- True
isA b -- True
isA c -- False
a == b  -- ошибка компиляции?
a == c  -- ошибка компиляции?
\end{minted}
\end{frame}

\begin{frame}[fragile]{Eq}
	\begin{itemize}
		\item Pattern Matching "--- конструкция на уровне языка.
		\item \t{==} "--- просто некоторая функция с таким названием.
		\item В C++ мы бы написали перегрузку функции/оператора.
		\item В Haskell пишем так:
\begin{minted}{haskell}
instance Eq IntList where
  Empty       == Empty       = True
  (Cons x xs) == (Cons y ys) = (x == y) && (xs == ys)
  _           == _           = False

a == b  -- True
a == c  -- False
b == c  -- False
Empty == Empty           -- True
Empty /= (Cons 1 Empty)  -- True, /= тоже работает
\end{minted}
	\end{itemize}
\end{frame}

\begin{frame}[fragile]{class Eq}
	\begin{itemize}
		\item \t{Eq} "--- это \textit{класс типов}, который описывает, что к типам можно применять определённые функции:
\begin{minted}{haskell}
class Eq a where
  (==) :: a -> a -> Bool
  (/=) :: a -> a -> Bool
\end{minted}
		\item Говорим, что тип \t{a} лежит в классе \t{Eq} тогда и только тогда, когда для него есть функции \t{(==)} и \t{(/=)}
		\item Класс типов "--- это такой <<интерфейс>> для типов.
		\item Некоторые функции требуют, чтобы параметры были в определённых классах:
\begin{minted}{haskell}
lookup :: Eq a => a -> [(a, b)] -> Maybe b
\end{minted}
		\item Слово \t{instance} на предыдущем слайде добавляло \t{IntList} в класс \t{Eq}.
		\item Не путать с классами объектов из ООП!
	\end{itemize}
\end{frame}

\begin{frame}[fragile]{Реализации по умолчанию}
	\begin{itemize}
		\item Для \t{IntList} мы реализовали только \t{==}, а \t{/=} получили автоматом.
		\item В классе можно указывать реализацию по умолчанию:
\begin{minted}{haskell}
class Eq a where
  (==) :: a -> a -> Bool
  (==) a b = not (a /= b)
  (/=) :: a -> a -> Bool
  (/=) a b = not (a == b)
\end{minted}
		\item Тогда \textit{минимальное полное определение} для \t{Eq} "--- это либо \t{==}, либо \t{/=}.
	\end{itemize}
\end{frame}

\subsection{Для параметризованных типов}
\begin{frame}[fragile]{Класс Eq для списков}
	\begin{itemize}
		\item Пусть есть свой класс для списков:
\begin{minted}{haskell}
data List a = Empty | Cons a (List a)
\end{minted}
		\item Разумно считать, что списки равны, если равны элементы:
\begin{minted}{haskell}
instance Eq (List a) where
  Empty       == Empty       = True
  (Cons x xs) == (Cons y ys) = (x == y) && (xs == ys)
  _ == _                     = False
\end{minted}
		\item Не скомпилируется, потому что элементы произвольного типа \t{a} нельзя сранивать.
		\item Надо добавить \textit{контекст} "--- сказать, что списки можно сравнивать только если можно сранивать элементы:
\begin{minted}{haskell}
instance Eq a => Eq (List a) where
\end{minted}
	\end{itemize}
\end{frame}

\begin{frame}[fragile]{Классы для структур данных}
	\begin{itemize}
		\item Иногда хочется указать, что структура данных обладает некоторым свойством:
\begin{minted}{haskell}
class Mappable f where
  map' :: (a -> b) -> f a -> f b
\end{minted}
		\item \t{Mappable} "--- что-то, содержащее элементы произвольного типа, к чему можно делать \t{map'}.
		\item Можно реализовать:
\begin{minted}{haskell}
instance Mappable List where
  map' _ Empty = Empty
  map' f (Cons x xs) = Cons (f x) (map' f xs)

map' (+1) (Cons 1 (Cons 10 Empty))
-- Cons 2 (Cons 11 Empty)
\end{minted}
	\end{itemize}
\end{frame}

\begin{frame}[fragile]{Functor}
	\begin{itemize}
		\item Аналог \t{Mappable} в Haskell называется \t{Functor}, в нём определена функция \t{fmap}.
		\item \t{fmap} должна удовлетворять некоторым аксиомам, но их компилятор сам не проверит:
\begin{minted}{haskell}
fmap id x == x
fmap (f . g) x == fmap f (fmap g x)
\end{minted}
		\item После реализации \t{Functor} наш новый тип можно использовать в том числе в старых функциях:
\begin{minted}{haskell}
a = (Cons 1 (Cons 2 (Cons 3 Empty)))
fmap (+1) [1,2,3]
fmap (+1) a
-- Встроенный оператор замены значений на константу
10 <$ [1,2,3]  -- [10,10,10]
10 <$ a
\end{minted}
	\end{itemize}
\end{frame}

\subsection{Прочие плюшки}
\begin{frame}[fragile]{Свои классы типов}
	Можно создать свой собственный класс:
\begin{minted}{haskell}
class HasSize a where
  size :: a -> Int

instance HasSize [a] where
  size xs = length xs

data Tree a = Nil | Node a (Tree a) (Tree a)
instance HasSize (Tree a) where
  size Nil = 0
  size (Node _ l r) = size l + size r
\end{minted}
	\begin{itemize}
		\item
			Можно определить даже для старых типов.
			В Java/C++ такое можно сделать только внешними перегруженными функциями.
		\item Новые функции можно использовать где угодно с любыми типами.
	\end{itemize}
\end{frame}

\begin{frame}[fragile]{Стандартные классы}
	\begin{itemize}
		\item \t{Show} "--- то, что можно вывести на экран.
		\item \t{Eq} "--- операторы \t{==} и \t{/=}.
		\item \t{Ord} "--- операторы \t{<}, \t{<=} и прочие.
		\item \t{Functor} "--- структура данных, на которой есть \t{map}.
		\item \t{Foldable} "--- структура данных, на которой есть \t{foldr} (по сути, умеет разворачиваться в список).
		\item Для первых трёх Haskell умеет сам генерировать адекватные реализации, если попросить:
\begin{minted}{haskell}
data List a = Empty | Cons a (List a)
            deriving (Show, Eq, Ord)
\end{minted}
		\item Порядок <<лексикографический>> (более ранний конструктор меньше).
	\end{itemize}
\end{frame}

\begin{frame}[fragile]{Зависимости классов}
	\begin{itemize}
		\item Как определить \t{Ord}? Хотим, чтобы тип класса \t{Ord} автоматически подходил везде, где нужен лишь \t{Eq}.
		\item Можно скопировать \t{==} и \t{/=} в \t{Ord}, но тогда:
			\begin{enumerate}
				\item Можно случайно реализовать и \t{Ord}, и \t{Eq}, причём по-разному.
				\item Получаем <<утиную типизацию>>: один класс вкладывается в другой, если есть функции с такими же названиями.
				\item Это нехорошо: придётся следить, что названия функций вообще нигде не пересекаются.
			\end{enumerate}
		\item Можно сказать, что \t{Ord} определяет только новые операторы, но тогда мы можем определить тип с \t{<}, но без \t{==}, что странно.
		\item Решение: класс \t{Ord} требует, чтобы тип также был в классе \t{Eq}:
\begin{minted}{haskell}
class Eq a => Ord a where
  (<), (<=), (>=), (>)  :: a -> a -> Bool
\end{minted}
		\item Если где-то пишем контекст \t{Ord a}, то \t{Eq a} появляется неявно.
		\item В частности, в реализациях по умолчанию в \t{Ord a} можно использовать \t{==} и \t{/=}.
	\end{itemize}
\end{frame}

\begin{frame}[fragile]{Автовывод типов}
\begin{minted}{haskell}
-- Ord a => a -> a -> a
max' a b = a == b

-- (Functor f, Eq a) => a -> f a -> f (Maybe a)
removeByValue x ys = fmap f ys
  where
    f y | x == y    = Nothing
        | otherwise = Just y
\end{minted}
	Если в файле не видно разных функций с одинаковым названием из разных классов, то компилятор может автоматически вывести ограничения на типы (контекст).
\end{frame}

\begin{frame}[fragile]{Резюме}
	\begin{itemize}
		\item Альтернатива классам типов "--- интерфейсы из ООП или перегрузки функций.
		\item Перегрузки функций не отражают связи между разными функциями (вроде \t{==} и \t{/=}).
		\item Интерфейсы из ООП \textit{обычно} надо определять в момент создания каждого типа (не добавить интерфейс к уже существующему).
		\item Интерфейсы из ООП \textit{обычно} не позволяют делать реализации по умолчанию "--- надо писать руками.
		\item Классы типов всё это позволяют.
		\item Компилятор умеет автоматически выводить нужный контекст.
		\item В Haskell типы классов используется везде, где есть хотя бы доля обобщаемости.
	\end{itemize}
\end{frame}

\begin{frame}[fragile]{Упражнение}
	\begin{enumerate}
		\item Пусть имеется следующее дерево со значениями в листьях и произвольным числом детей:
\begin{minted}{haskell}
data Tree a = Leaf a | Node [a] deriving Show
\end{minted}
		\item Добавьте Tree в класс \t{Eq}. Подсказка: списки там уже есть.
		\item Пусть есть такой класс для структур, в которых можно развернуть порядок элементов:
\begin{minted}{haskell}
class Reversible f where
   reverse :: f a -> f a
\end{minted}
		\item Добавьте списки в этот класс:
\begin{minted}{haskell}
class Reversible [] where
\end{minted}
		\item Добавьте дерево в этот класс:
\begin{minted}{haskell}
class Reversible Tree where
\end{minted}
	\end{enumerate}
\end{frame}
